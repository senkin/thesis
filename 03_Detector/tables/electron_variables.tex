\begin{longtable}{@{}p{.2\textwidth} p{.75\textwidth}@{}}
  \caption{Variables used in electron identification algorithms} \label{tab:electron_ID_variables} \\
  \toprule
  \textbf{Variable} & \textbf{Description} \\    
  \midrule
  \multicolumn{2}{c}{\textit{Track quality variables}} \\
  \midrule
  \pt & Transverse momentum of the GSF track \\
  %\midrule
  $\eta$ & Pseudorapidity of the GSF track \\
  %\midrule
  GSF $\sigma_{p_\text{T}}/p_\text{T}$ & Transverse momentum resolution of the GSF track\\
  %\midrule
  \#hits$_\text{KF}$ & Number of reconstructed KF track hits\\
  %\midrule
  $\chi^2_\textrm{GSF}$ and $\chi^2_\textrm{KF}$ & GSF and KF goodness-of-fits\\
  \midrule

  \multicolumn{2}{c}{\textit{ECAL shower variables}} \\
  \midrule
  $\sigma^2_{i\eta, i\eta}$ & Cluster shape variable that gives a measure of the width of the cluster in $\eta$, using the distribution of energy in a $5\times5$ block of crystals around the seed crystal (the one with the highest energy) \autocite{electron_reconstruction}:
  $\sigma^2_{i\eta, i\eta} = \sum_{5\times5 \text{ crystals}} \left(\eta_i - \eta_\text{seed cluster}\right)^2 E_i / E_{\text{seed cluster}}$ \\
  %\midrule
  $\sigma^2_{i\phi, i\phi}$ & Cluster shape in $\phi$\\
  %\midrule
  $\eta_\text{SC}$ ($\phi_\text{SC}$) & Width of the super-cluster in $\eta$ ($\phi$)\\
  %\midrule
  $1-E_{1\times 5}/E_{5\times 5}$ & $E_{1\times 5}$ is the energy in the central $1\times 5$ strip of the
  $5\times 5$ electron cluster, and $E_{5\times 5}$ is its total energy\\
  %\midrule
  $E_{3\times 3}/E_\text{SC, raw}$ & Ratio of the energy of a cluster of $3\times 3$ to the
  uncorrected (raw) energy of the super-cluster\\
  %\midrule
  $E_\text{PS}/E_\text{SC, raw}$ & Ratio of the energy in the preshower detector to the raw
  super-cluster energy (only in the endcap region)\\ 
  \midrule

  \multicolumn{2}{c}{\textit{Longitudinal shower shape variables}} \\
  \midrule
  $H/E$ & Ratio of the hadronic energy associated with the electron candidate to the super-cluster energy. The hadronic energy is found by summing the HCAL towers in a cone of radius $\Delta R = 0.15$, centred at the super-cluster position\\
  %\midrule
  $H/(H+E_\text{e})$ & Hadron fraction of the shower, where $H$ is the energy of the hadron cluster linked to
  the GSF track\\
  \midrule
  \newpage
  \caption*{Table~\ref{tab:electron_ID_variables}: Variables used in electron identification algorithms (continued)} \\
  \midrule

  \multicolumn{2}{c}{\textit{Track/super-cluster matching variables}} \\
  \midrule
  $\Delta\eta_{\text{in}}$ ($\Delta\phi_{\text{in}}$) & Distance in $\eta$ ($\phi$) between the super-cluster
  position and the extrapolated track position \\
  %\midrule
  $\Delta\eta_{\text{vtx}}$ ($\Delta\phi_{\text{vtx}}$) & Distance in $\eta$ ($\phi$) between the super-cluster
  position and the position of the GSF track at vertex \\
  %\midrule
  $(E_\text{e}+\sum E_\gamma)/p_\text{in}$ & Ratio of the super-cluster energy to the inner track\newline
  momentum\\
  %\midrule
  $E_\text{e}/p_\text{out}$ & Ratio of the electron cluster energy to the outer track\newline momentum\\
  %\midrule
  $1/E_\text{SC} - 1/p_\text{T}$ & Difference between inverse super-cluster energy and inverse
  track momentum\\
  \midrule

  \multicolumn{2}{c}{\textit{Bremsstrahlung variables}} \\
  \midrule
  $f_\text{brem}$ & Measured bremsstrahlung fraction, defined as:
  $f_\text{brem} = (p_\text{in}-p_\text{out})/p_\text{in}$, where $p_\text{in}$ is the initial track momentum at the vertex and $p_\text{out}$ is the track momentum at the last hit\\
  %\midrule
  $\sum E_\gamma/(p_\text{in}-p_\text{out})$ & Ratio between the Bremsstrahlung photon energy as measured by
  ECAL and by the tracker\\
  %\midrule
  $EarlyBrem$ & Flag of $(E_\text{e}+\sum E_\gamma)>p_\text{in}$ inequality, corresponding to an electron
  emitting an ``early'' Bremsstrahlung photon, i.e.\ before it has crossed at least three tracker layers\\
  %\midrule
  $LateBrem$ & Flag of $E_\text{e}>p_\text{out}$ inequality, corresponding to an electron emitting a ``late''
  Bremsstrahlung electron, when the ECAL clustering is not able to disentangle the overlapping electron and photon
  showers\\

  \bottomrule
\end{longtable}
