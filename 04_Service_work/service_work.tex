%!TEX root = ../thesis.tex

\chapter{High level trigger development for Top Physics}
\label{c:service_work}
\ifpdf
    \graphicspath{{04_Service_work/plots/}}
\else
    \graphicspath{{04_Service_work/plots/EPS/}{04_Service_work/plots/}}
\fi

The LHC is often referred to as a top quark factory, producing a \ttbar pair nearly each second of its nominal
operation. While the production rate of \SI{\approx1}{\Hz} seems manageable in terms of recording the data, it is
significantly complicated by background processes with similar signatures occuring at much higher rates.

The trigger is the starting point of any physics event selection process, and therefore is clearly important for any
physics analysis. As it was mentioned in Section~\ref{ss:trigger_daq}, the CMS L1 trigger rate is limited to
\SI{\sim100}{\kilo\hertz}. In order to meet the data recording constraints of approximately
\SI{300}{\mega\byte\per\second}, this rate is further reduced down to \SI{\sim300}{\Hz}, which is done by the HLT
system. The total rate budget has to be shared between various physics analysis groups (e.g.\ Top, Higgs, Exotica,
etc.). Corresponding allocations are determined by CMS trigger coordination according to CMS physics goals, and can be a
matter of serious debate.

During the LHC operation in 2011 and 2012 under conditions of gradual increase of instantaneous luminosity and pile-up,
but very limited rate budget, trigger developers constantly faced the challenge of finding the best compromise between
growing rates and maintaining reasonable signal efficiency. While the simplest approach is tightening the cuts on
physics quantities like lepton or jet transverse momenta, it is not favourable since it lowers the number of stored
signal events and decreases the phase space which is crucial for new physics searches as well as Standard Model
precision measurements. Therefore, development of more efficient algorithms allowing to keep high level of acceptance
for signal events, whilst effectively rejecting background events, is the most preferrable solution. This can often be
achieved by increasing the level of approximation of the online (HLT) object reconstruction, making the algorithms
closer to their sophisticated offline counterparts. However, it leads to a higher execution time, which is limited by
the computing resources available for HLT reconstruction. Hence, CPU timing is another major constraint faced by the HLT
developers.

This chapter covers my contribution to development and maintenance of High-Level Triggers for top physics with
semileptonic signature, where one of the W bosons decays into an electron and a neutrino. Trigger efficiency
measurement, CPU timing studies, validation of jet energy corrections and pile-up subtraction applied at the HLT level
are discussed in relevant sections of this chapter.

\section{Level-1 triggers}
The L1 trigger budget of \SI{\sim100}{\kilo\hertz} is shared between several L1 trigger ``seeds'', corresponding to
different physics objects being present in the event.

\section{High-level triggers for top physics}

\section{Trigger efficiency measurement}

\section{CPU timing studies}

\section{Summary}


% ------------------------------------------------------------------------


%%% Local Variables: 
%%% mode: latex
%%% TeX-master: "../thesis"
%%% End: 
