%!TEX root = ../thesis.tex

%%% Thesis Introduction --------------------------------------------------
\chapter{Introduction}
\label{c:intro}
\ifpdf
    \graphicspath{{01_Introduction/plots/}}
\else
    \graphicspath{{01_Introduction/plots/EPS/}{01_Introduction/plots/}}
\fi

The beginning of the 21st century has been a truly exciting time in particle physics. First LHC collisions in 2009
marked the start of the new era, which only three years later brought the discovery of the Higgs boson, the last missing
piece of the Standard Model. The hopes are high that it was not the last discovery of this era, and new physics is
hiding around the corner at the energies accessible to the LHC.

While searches for the new physics are very exciting, one should not underestimate the importance of precision
measurements of the Standard Model. Even after the Higgs boson discovery, top quark physics has remained in the
priorities of the LHC physics programme. One of the main reasons is its importance as a primary background to many new
physics scenarios beyond the Standard Model. Moreover, it is still not understood why the top quark Yukawa coupling is
so close to unity, which implies that severe fine tuning of the Higgs mass happens mostly due to the top quark. Many
extensions of the Standard Model offer solutions to this hierarchy problem by extending the top quark sector and
introducing more degrees of freedom, which are expected to cause deviations from the Standard Model predictions in top
quark-related observables. Therefore, precise measurements of the top quark properties are of high importance.

The top quark mass is a crucial fundamental parameter of the Standard Model. The mass analysis presented in this thesis
contributes to the most precise single measurement of the top quark mass up to date. While the systematic uncertainty
due to the jet energy scale (JES) in author's work is significantly larger than that of the published CMS measurement,
the analysis serves as an important cross-check of the mass extraction technique and the kinematic fitting procedure.
The mentioned JES systematic uncertainty is mitigated in the published measurement via the \textit{in situ} measurement
of the JES and the top quark mass in a joint likelihood fit. This thesis shows the consistency of these measurements
within uncertainties, as well as with the first world combination of the top quark mass measurements.

The LHC has proven to be a top quark factory, with an abundance of statistics pushing the limits of precision
calculations to the level of uncertainties of theoretical predictions. All these data need to be carefully selected, as
recording the top quark events is significantly complicated by background processes with similar signatures occurring at
much higher rates. Therefore, an efficient and manageable trigger is a crucial part of any top quark analysis. In this
thesis, work on the high-level triggers used for selection of top quark events with semileptonic signature, particularly
with an electron and jets in the final state, is presented.

For the first time in history of particle physics, the abundance of top quark pair (\ttbar) events at the LHC gives an
opportunity to measure the \ttbar differential cross section with respect to various quantities. Both the CMS and ATLAS
collaborations have published results on such measurements with respect to top quark-related variables. In this thesis,
the \ttbar differential cross section is measured with respect to the event-level distributions, which do not require a
kinematic reconstruction of the \ttbar decay. The absence of systematic uncertainties associated with the kinematic
reconstruction is the main advantage of these measurements, which allows more detailed comparison of the data and
predictions by different Monte Carlo generators. Furthermore, new physics could reveal itself in the tails of these
event-level distributions. For instance, an associated production of a \ttbar pair with a new resonance
($\ttbar+\mathrm{X}$) decaying invisibly may show up in the tail of the missing transverse energy distribution.

The work on the top quark mass analysis using the \SI{7}{\TeV} LHC data was performed at CERN in collaboration with
Martijn Mulders and Enrique Palencia. The author's main contribution was to the electron side of the analysis,
particularly all the analysis steps up to setting up the kinematic fit and obtaining the fitted information which was
used in the mass extraction technique. The high-level triggers used in the analysis were developed by the author in
collaboration with Lukasz Kreczko and St\'{e}phanie Beauceron.

The differential cross section analysis based on both \SI{7}{\TeV} and \SI{8}{\TeV} LHC datasets was done in close
collaboration with Lukasz Kreczko, Jeson Jacon and Phil Symonds under supervision of Greg Heath and Joel Goldstein. On
the technical side of the analysis, the author mainly focused on developing the C\texttt{++}/Python-based analysis
framework (Bristol Analysis Software) and ensuring the latest and most precise physics object definitions were used;
producing the local n-tuples; programming Python scripts in conjunction with ROOT/PyROOT software to extract the
differential cross section and produce final results. On the physics side, the main contribution was also to the
electron channel, particularly the missing transverse energy measurement, data-driven QCD estimation, evaluation of
systematic uncertainties and implementation of the SVD unfolding.

The thesis is structured as follows. Some theoretical background on the Standard Model and top quark physics is given in
Chapter~\ref{c:theory}. Then the LHC and the CMS detector, as well as the relevant aspects of Monte Carlo simulation and
object reconstruction are discussed in Chapter~\ref{c:detector}. The author's contribution to the CMS high-level
trigger development is detailed in Chapter~\ref{c:service_work}. Afterwards, two major analysis chapters are presented:
the top quark mass measurement in Chapter~\ref{c:top_mass_analysis}, and the top quark pair differential cross section
measurement in Chapter~\ref{c:xsection_analysis}. Finally, conclusions and outlook are given in the last chapter.

\section*{Conventions}
In this thesis, the natural units convention is adopted:
\begin{equation*}
\hslash = c = 1,
\end{equation*}
therefore, the same units (electronvolts, \si{\eV}) are used to denote momentum, energy and mass quantities. Typically,
these would be \si{\GeV} units (\SI{1}{\GeV} = \SI{e9}{\eV}), due to the energies accessible at the LHC. The coordinate
system used to describe the geometry of the detector and event kinematics is explained in Section~\ref{s:CMS}.


%%% ----------------------------------------------------------------------


%%% Local Variables: 
%%% mode: latex
%%% TeX-master: "../thesis"
%%% End: 
