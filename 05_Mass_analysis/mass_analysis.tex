%!TEX root = ../thesis.tex

\chapter{Top Quark mass measurement}
\label{c:top_mass_analysis}
\ifpdf
    \graphicspath{{05_Mass_analysis/plots/}}
\else
    \graphicspath{{05_Mass_analysis/plots/EPS/}{05_Mass_analysis/plots/}}
\fi

In this chapter the top mass measurement using 2011 LHC data recorded by the CMS detector at a centre of mass energy of
$\sqrt s =$ \SI{7}{\TeV} is presented. This analysis was a cross-check of the official CMS top mass measurement at
\SI{7}{\TeV} in lepton plus jets channel published in 2012 \cite{top_mass_ljets_CMS}. Only the electron channel is
described here, as the muon side of this analysis was performed by a different group at CERN, although in close
collaboration.

The mass extraction method used in this analysis is essentially the same to one used in the CMS measurement of the mass
difference between top and antitop quarks \cite{mass_difference_CMS}.

%talk about JES limiting factor, why the analysis is a cross check, why is it important, cite the main CMS paper, a
%couple of words about the ideogram method... anything else? theory motivation should be in the first chapter,
%reference it.

\section{Data and Simulation}
\label{s_top_mass:data_and_simulation}

\subsection{Data}
\label{ss_top_mass:data}
This analysis uses the full 2011 data recorded by the CMS detector with a total integrated luminosity of
\SI{5.0 \pm 0.1}{\fbinv}. As only the electron channel is covered by this particular analysis, all data was pre-selected
by single electron plus jets high level triggers, described in detail in Chapter \ref{c:service_work}.

% \begin{itemize}
%   \item electron + 3 (PF)jets
%   \item isolated electron + 3 (PF)jets
% \end{itemize}

% \begin{itemize}
% 	\item /ElectronHad/Run2011A-May10ReReco-v1/AOD
% 	\item /ElectronHad/Run2011A-PromptReco-v4/AOD
% 	\item /ElectronHad/Run2011A-05Aug2011-v1/AOD
% 	\item /ElectronHad/Run2011A-PromptReco-v6/AOD
% 	\item /ElectronHad/Run2011B-PromptReco-v1/AOD
% \end{itemize}

\subsection{Simulation of signal and background processes}
\label{ss_top_mass:signal_and_background}

To develop and test any analysis technique, simulated events from Monte Carlo (MC) generators are inevitably needed. In
this work the \ttbar signal and W+Jets background events were generated with the MadGraph matrix element generator
\cite{MadGraph}, Pythia parton showering \cite{Pythia} and a full GEANT4-based detector simulation \cite{GEANT4}. The
single top background was simulated using Powheg \cite{POWHEG}. Let us briefly describe these generators.

% The \ttbar signal is available with 9 different generated top quark masses and
% systematic variations. Pile up reweighting is used to match the simulation to the vertex multiplicity in data. The
% MC/data differences of b-tagging efficiencies and trigger efficiencies are taken into account by additional MC scale
% factors. The jet energy resolution is scaled to match the resolution in data.

% Signal:
% \begin{itemize}
% \item {\footnotesize /TTJets\_TuneZ2\_mass161\_5\_7TeV-madgraph-tauola/Fall11-PU\_S6\_START42\_V14B-v3/AODSIM}
% \item {\footnotesize /TTJets\_TuneZ2\_mass163\_5\_7TeV-madgraph-tauola/Summer11-PU\_S4\_START42\_V11-v3/AODSIM}
% \item {\footnotesize /TTJets\_TuneZ2\_mass166\_5\_7TeV-madgraph-tauola/Summer11-PU\_S4\_START42\_V11-v3/AODSIM}
% \item {\footnotesize /TTJets\_TuneZ2\_mass169\_5\_7TeV-madgraph-tauola/Summer11-PU\_S4\_START42\_V11-v3/AODSIM}
% \item {\footnotesize /TTJets\_TuneZ2\_7TeV-madgraph-tauola/Fall11-PU\_S6\_START42\_V14B-v2/AODSIM}
% \item {\footnotesize /TTJets\_TuneZ2\_mass175\_5\_7TeV-madgraph-tauola/Summer11-PU\_S4\_START42\_V11-v3/AODSIM}
% \item {\footnotesize /TTJets\_TuneZ2\_mass178\_5\_7TeV-madgraph-tauola/Summer11-PU\_S4\_START42\_V11-v3/AODSIM}
% \item {\footnotesize /TTJets\_TuneZ2\_mass181\_5\_7TeV-madgraph-tauola/Summer11-PU\_S4\_START42\_V11-v3/AODSIM}
% \item {\footnotesize /TTJets\_TuneZ2\_mass184\_5\_7TeV-madgraph-tauola/Fall11-PU\_S6\_START42\_V14B-v1/AODSIM}
% \end{itemize}
% Background:
% \begin{itemize}
% \item {\footnotesize /QCD\_Pt-20\_MuEnrichedPt-15\_TuneZ2\_7TeV-pythia6/Fall11-PU\_S6\_START42\_V14B-v1/AODSIM}
% \item {\footnotesize /QCD\_Pt-20to30\_EMEnriched\_TuneZ2\_7TeV-pythia6/Fall11-PU\_S6\_START42\_V14B-v1/AODSIM}
% \item {\footnotesize /QCD\_Pt-30to80\_EMEnriched\_TuneZ2\_7TeV-pythia/Fall11-PU\_S6\_START42\_V14B-v1/AODSIM}
% \item {\footnotesize /QCD\_Pt-80to170\_EMEnriched\_TuneZ2\_7TeV-pythia6/Fall11-PU\_S6\_START42\_V14B-v2/AODSIM}
% \item {\footnotesize /QCD\_Pt-20to30\_BCtoE\_TuneZ2\_7TeV-pythia6/Fall11-PU\_S6\_START42\_V14B-v1/AODSIM}
% \item {\footnotesize /QCD\_Pt-30to80\_BCtoE\_TuneZ2\_7TeV-pythia6/Fall11-PU\_S6\_START42\_V14B-v1/AODSIM}
% \item {\footnotesize /QCD\_Pt-80to170\_BCtoE\_TuneZ2\_7TeV-pythia/Fall11-PU\_S6\_START42\_V14B-v1/AODSIM}
% \item {\footnotesize /WJetsToLNu\_TuneZ2\_7TeV-madgraph-tauola/Fall11-PU\_S6\_START42\_V14B-v1/AODSIM}
% \item {\footnotesize /DYJetsToLL\_TuneZ2\_M-50\_7TeV-madgraph-tauola/Fall11-PU\_S6\_START42\_V14B-v1/AODSIM}
% \item {\footnotesize /Tbar\_TuneZ2\_s-channel\_7TeV-powheg-tauola/Summer11-PU\_S4\_START42\_V11-v1/AODSIM}
% \item {\footnotesize /Tbar\_TuneZ2\_t-channel\_7TeV-powheg-tauola/Summer11-PU\_S4\_START42\_V11-v1/AODSIM}
% \item {\footnotesize /Tbar\_TuneZ2\_tW-channel-DR\_7TeV-powheg-tauola/Summer11-PU\_S4\_START42\_V11-v1/AODSIM}
% \item {\footnotesize /T\_TuneZ2\_s-channel\_7TeV-powheg-tauola/Summer11-PU\_S4\_START42\_V11-v1/AODSIM}
% \item {\footnotesize /T\_TuneZ2\_t-channel\_7TeV-powheg-tauola/Summer11-PU\_S4\_START42\_V11-v1/AODSIM}
% \item {\footnotesize /T\_TuneZ2\_tW-channel-DR\_7TeV-powheg-tauola/Summer11-PU\_S4\_START42\_V11-v1/AODSIM}
% \end{itemize}
% Systematic variations:
% \begin{itemize}
% \item {\footnotesize (/TTjets\_TuneZ2\_matchingdown\_7TeV-madgraph-tauola/Fall11-PU\_S4\_START42\_V11-v1/AODSIM)}
% \item {\footnotesize /TTjets\_TuneZ2\_matchingup\_7TeV-madgraph-tauola/Fall11-PU\_S6\_START42\_V14B-v2/AODSIM}
% \item {\footnotesize /TTjets\_TuneZ2\_scaledown\_7TeV-madgraph-tauola/Fall11-PU\_S6\_START42\_V14B-v2/AODSIM}
% \item {\footnotesize /TTjets\_TuneZ2\_scaleup\_7TeV-madgraph-tauola/Fall11-PU\_S6\_START42\_V14B-v1/AODSIM}
% \item {\footnotesize /TTJets\_TuneP11\_7TeV-madgraph-tauola/Fall11-PU\_S6\_START42\_V14B-v1/AODSIM}
% \item {\footnotesize (/TTJets\_TuneP11noCR\_7TeV-madgraph-tauola/Fall11-PU\_S6\_START42\_V14B-v1/AODSIM)}
% \end{itemize}

\subsection{Monte Carlo generators}
\label{ss_top_mass:MC_generators}

In order to simulate the previously described signal and background processes several Monte Carlo (MC) event generators
are used. Given the process the event generators produce the long chain needed for a event: production of the main
process; particle decays; boson radiation (\Z, \photon, \cPg); and the hadronisation of quarks and gluons. Each
generator is optimised in one or more aspects of this chain and is therefore either used in conjunction with another
generator (signal generation with \MADGRAPH + \PYTHIA) or used for modelling the systematic uncertainty on the theory
due to different prescriptions (\POWHEG + \PYTHIA, \MCATNLO). As a specialist in radiation and hadronisation \PYTHIA is
either used by itself (QCD \multijet events) or takes over from the other generators at this step.

\subsubsection*{MadGraph}

\MADGRAPH \cite{MadGraph} is a matrix-element Monte Carlo event generator. For any renormalisable Lagrangian based model
it produces all possible Feynman diagrams and automatically generates its matrix elements at the tree-level, performing
the integration over all phase-space. This calculation is then used to produce the \xsect of various processes and
subprocesses as well as partons and kinematics of an event, including decays that are described using spin correlations.
The partons from matrix element calculations are then matched to parton showers from hadronisation of quarks and gluons,
which is simulated in \PYTHIA (see below) along with the fragmentation of initial protons and the soft scattering of
underlying event (see Section~\ref{sss:JEC}). The matching is done according to so-called MLM prescription \cite{MLM} if
a parton-jet pair satisfies a certain $\Delta R$ separation criteria. If no or more than one matched jets are found,
events are rejected. There is also a certain transverse energy threshold requirement for the partons to be considered in
the matching. Default values depend on the process and are set to be \SI{20}{\GeV} for \ttbar events and
\SI{10}{\GeV} for \WpJets and \ZpJets events.

%An additional package (\TAUOLA \cite{TAUOLA}) is used for tau lepton decay description.

\subsubsection*{PYTHIA} 

\PYTHIA \cite{Pythia,Pythia6.4} is perhaps the most widely used MC generator in high energy physics. It is a standard
tool used for simulation of quark and gluon hadronisation, multi-particle production, beam remnants, initial protons
fragmentation, etc. Therefore \PYTHIA is used to generate QCD multi-jet production and underlying event on top of other
generators providing partons from hard processes.

\subsubsection*{POWHEG} 

\POWHEG (Positive Weight Hardest Emission Generator) \cite{POWHEG} was proposed to overcome the problem of negative
event weights which arise in the \MCATNLO method when matching NLO QCD computations with parton showers. Compared to
other generators \POWHEG generates the hardest radiation first. This is done with a technique that yields only
positive-weighted events using the exact NLO matrix elements.

\subsection{Factorisation Scale and Matching Threshold}
\label{ss:factorisation_and_matching}
%https://twiki.cern.ch/twiki/bin/view/CMS/MadGraphStandardModel

As previously described, the matching between the partons from the matrix-element calculations and the parton showers is
performed only for partons which exceed a transverse energy threshold. The systematic uncertainty resulting from the
choice of this threshold is estimated by using samples that have been produced with half and double the chosen
threshold. A similar procedure is deployed for the choice of the factorisation scale, which determines the scale at
which $\alpS$ is evaluated.
A summary of the used values for the uncertainty calculation is shown in table.


\subsection{Detector simulation with GEANT}
After the full physics simulation the last important step of the simulation process is the simulation of the detector
and its interaction with the generated particles. This process is performed in the detector simulation package
\GEANTfour (GEometry ANd Tracking) \cite{GEANT4}. \GEANTfour recreates the geometry of the detector, describes the
material in the detector and particle interactions within it such as tracking and detector response.

\subsection{Monte Carlo samples}
The simulated MC events are generated at the center of mass energy of \SI{7}{\TeV}, using the CTEQ6L Parton Distribution
Functions (PDF) \cite{cteq}. They are produced separately for each process. The \zp signal as well as the main
background processes are simulated in \MADGRAPH. Signal samples are produced with various masses and widths of the \zp
boson (see table \ref{tab:zprime_signal_mc}). To extend the available statistics for the \WpJets sample four additional
\WpJets samples are included. These samples are generated in exclusive jet multiplicity bins: \W boson production plus
with one/two/three and at least four jets.

The QCD background samples are generated in \PYTHIA while the \photon + jets samples are generated with \MADGRAPH (table
\ref{tab:zprime_background_qcd}). Since only a small fraction of the QCD \multijet events is of use in this analysis,
samples with generator level cuts are used to enhance the available statistics after the selection. These samples either
contain jets with high electromagnetic content (EM-enriched) or contain heavy flavour quarks (b or c) which decay into
electron plus neutrino ($\cPqb/\cPqc \rightarrow \cPq e\nu$). These two sets of QCD samples are generated in different
bins of $\hat\pt$ where most events surviving the selection come from the high bins. $\hat\pt$ denotes the transverse
momentum of the partons produced. The \photon plus jets samples are generated in different bins of \HT, defined as the
sum of transverse momenta of all particles.

\begin{table}[!htbp]
\centering
\begin{tabular}{|l|l|r|r|r|}
\toprule
Process & Generator & $\sigma$ (\pb) & \# events & $\int\lumi dt$ (\fbinv)\\
\midrule
\ttjets & \MADGRAPH & & & \\
\hspace{5 mm}\mtop = \SI{161.5}{\GeV} & & 157.5 & 1620072 & 10.3\\
\hspace{5 mm}\mtop = \SI{163.5}{\GeV} & & 157.5 & 1633197 & 10.4\\
\hspace{5 mm}\mtop = \SI{166.5}{\GeV} & & 157.5 & 1669034 & 10.6\\
\hspace{5 mm}\mtop = \SI{169.5}{\GeV} & & 157.5 & 1606570 & 10.2\\
\hspace{5 mm}\mtop = \SI{172.5}{\GeV} & & 157.5 & 7490162 & 47.6\\
\hspace{5 mm}\mtop = \SI{175.5}{\GeV} & & 157.5 & 1538301 & 9.8\\
\hspace{5 mm}\mtop = \SI{178.5}{\GeV} & & 157.5 & 1648519 & 10.5\\
\hspace{5 mm}\mtop = \SI{181.5}{\GeV} & & 157.5 & 1665350 & 10.6\\
\hspace{5 mm}\mtop = \SI{184.5}{\GeV} & & 157.5 & 1671859 & 10.6\\
\midrule
\WpJets ($\W \rightarrow l\nu$) & \MADGRAPH & 31314 & 81345381 & 2.6 \\
% \hspace{5 mm}\W + 1 jet & & 4480 & 76051609 & 17.0 \\
% \hspace{5 mm}\W + 2 jet & & 1674 & 25400546 & 15.2 \\
% \hspace{5 mm}\W + 3 jet & & 484.7 & 7541595 & 15.6 \\
% \hspace{5 mm}\W + 4 jet & & 211.7 & 13133738 & 62.0 \\
\midrule
$\Z/\gamma^* \rightarrow l^+l^- $ + jets, $m(ll) > \SI{50}{\GeV}$ & \MADGRAPH & 3048 & 36222153 & 11.9 \\
\midrule
Single top & \POWHEG & & & \\
\hspace{5 mm} top $t$-channel & & 42.6 & 3814228 & 89.5 \\
\hspace{5 mm} anti-top $t$-channel & & 22.0 & 1944822 & 88.4 \\
\hspace{5 mm} top $s$-channel & & 2.72 & 259971 & 92.6 \\
\hspace{5 mm} anti-top $s$-channel & & 1.49 & 137980 & 92.6 \\
\hspace{5 mm} top $tW$-channel & & 5.3 & 814390 & 153.7 \\
\hspace{5 mm} anti-top $tW$-channel & & 5.3 & 809984 & 152.8 \\
\bottomrule
\end{tabular}
\caption{Signal and background Monte Carlo samples with cross sections at $\sqrt s =
\SI{7}{\TeV}$, numbers of generated events and corresponding integrated
luminosities.}
\label{tab:top_mass_mc_samples}
\end{table}

% \begin{table}[!hbth] \centering
% \begin{tabular}{lrrr}
% \toprule
% process & $\sigma$ (\pb) & \# events & $\int\lumi dt$ \fbinv\\
% \midrule
% \WpJets ($\W \rightarrow l\nu$) & 31314 & 81345381 & 2.6 \\
% \W + 1 jet ($\W \rightarrow l\nu$) & 4480 & 76051609 & 17.0 \\
% \W + 2 jet ($\W \rightarrow l\nu$) & 1674 & 25400546& 15.2 \\
% \W + 3 jet ($\W \rightarrow l\nu$) & 484.7 & 7541595 & 15.6 \\
% \W + 4 jet ($\W \rightarrow l\nu$) & 211.7 & 13133738 & 62.0 \\
% $\Z/\gamma^* \rightarrow l^+l^-, m(ll) > \SI{50}{\GeV}$ + jets & 3048 & 36222153 & 11.9 \\
% Single top t-channel & 42.6 & 3814228 & 89.5 \\
% Single anti-top t-channel & 22.0 & 1944822 & 88.4 \\
% Single top s-channel & 2.72 & 259971 & 92.6 \\
% Single anti-top s-channel & 1.49 & 137980 & 92.6 \\
% Single top tW-channel & 5.3 & 814390 & 153.7 \\
% Single anti-top tW-channel & 5.3 & 809984 & 152.8 \\
% \bottomrule
% \end{tabular}
% \caption{Background Monte Carlo samples, their cross sections at a \CoM energy
% of \SI{7}{\TeV}, numbers of generated events and corresponding integrated
% luminosities.}
% \label{tab:top_mass_background_mc}
% \end{table}

\begin{table}[!htbp] 
\centering
\resizebox{\textwidth}{!}{
\begin{tabular}{|l|l|r|r|r|r|}
\toprule
Process & Generator & $\sigma$ (\pb) & filter efficiency & \# events & $\int\lumi dt$ (\fbinv)\\
\midrule
QCD ($e/\gamma$ enriched)  & \PYTHIA & & & & \\
\hspace{5 mm}\SIrange[range-phrase = $~<\pthat<~$]{20}{30}{\GeV} & & \num{2.355d8} & \num{7.3d-3} & 34720808 & \num{2.0d-2} \\
\hspace{5 mm}\SIrange[range-phrase = $~<\pthat<~$]{30}{80}{\GeV}  & & \num{5.93d7}& \num{0.059} &  70375915 & \num{2.0d-2} \\
\hspace{5 mm}\SIrange[range-phrase = $~<\pthat<~$]{80}{170}{\GeV} & & \num{9.06d5} & \num{0.148} &  8150669& \num{6.1d-2} \\
\midrule
QCD ($\cPqb/\cPqc \rightarrow e\nu$) & \PYTHIA & & & & \\
\hspace{5 mm}\SIrange[range-phrase = $~<\pthat<~$]{20}{30}{\GeV} & & \num{2.355d8} & \num{4.6d-4} & 2002588 & \num{1.8d-2} \\
\hspace{5 mm}\SIrange[range-phrase = $~<\pthat<~$]{30}{80}{\GeV} & & \num{5.93d7}& \num{2.34d-3} & 2030030 & \num{1.5d-2} \\
\hspace{5 mm}\SIrange[range-phrase = $~<\pthat<~$]{80}{170}{\GeV} & & \num{9.06d5} & \num{0.0104} & 1082690 & \num{0.1} \\
\midrule
$\gamma$ + jets & \MADGRAPH & & & & \\
\hspace{5 mm}\SIrange[range-phrase = $~<\HT<~$]{40}{100}{\GeV} & & \num{23620} & 1 & 12730863 & \num{0.5} \\
\hspace{5 mm}\SIrange[range-phrase = $~<\HT<~$]{100}{200}{\GeV} & & \num{3476} & 1 & 1536287 & \num{0.4} \\
\hspace{5 mm}$\HT >$ \SI{200}{\GeV} & & \num{485} & 1 & 9377168 & \num{19.3} \\
\bottomrule
\end{tabular}
}
\caption{QCD multi-jet background and $\gamma$ + jets MC samples with cross sections at $\sqrt s =
\SI{7}{\TeV}$, numbers of generated events and corresponding integrated luminosities.
% EM-enriched samples are preselected to include jets with higher electromagnetic content;
% $\cPqb/\cPqc \rightarrow e\nu$ samples are preselected to include leptonic ($e\nu$) in-flight decays of b- and c-quarks.
}
\label{tab:top_mass_background_qcd}
\end{table}

\begin{table}[!htbp]
\centering
\begin{tabular}{|l|l|r|r|r|}
\toprule
Process & Generator & $\sigma$ (\pb) & \# events & $\int\lumi dt$ (\fbinv)\\
\midrule
\ttjets & \MADGRAPH & & & \\
\hspace{5 mm}$0.5~\times$ matching threshold & & 157.5 & 1607808& 10.2 \\
\hspace{5 mm}$2~\times$ matching threshold  & & 157.5 & 4029823& 25.5 \\
\hspace{5 mm}$0.5\times Q$  & & 157.5 & 4004587 & 25.4 \\
\hspace{5 mm}$2\times Q$ & & 157.5 & 3696269 & 23.4 \\
\midrule
\WpJets ($\W \rightarrow l\nu$) & \MADGRAPH & & & \\
\hspace{5 mm}$0.5~\times$ matching threshold & & 29690 & 9956679 & 0.3 \\
\hspace{5 mm}$2~\times$ matching threshold & & 30290 & 10461655 & 0.3 \\
\hspace{5 mm}$0.5 \times Q$ & & 33300 &10092532 & 0.3 \\
\hspace{5 mm}$2 \times Q$ & & 32000 &9784907 &0.3 \\
\midrule
\ZpJets ($\Z \rightarrow ll$) & \MADGRAPH & & & \\
\hspace{5 mm}$0.5~\times$ matching threshold & & 2888 & 1614808& 0.6 \\
\hspace{5 mm}$2~\times$ matching threshold & & 2915 & 1641121 & 0.6 \\
\hspace{5 mm}$0.5 \times Q$ & & 3312 & 1658855 & 0.5 \\
\hspace{5 mm}$2 \times Q$ & & 2954 & 1592742 & 0.5 \\
\bottomrule
\end{tabular}
\caption{Systematic MC samples with cross sections at $\sqrt s =
\SI{7}{\TeV}$, numbers of generated events and corresponding
integrated luminosities. Factorisation scale $Q$ and matching threshold
systematic uncertainties (see Section \ref{sss_top_mass:matching_and_factorisation}) are
estimated with variations of \ttjets, \WpJets and \ZpJets samples.}
\label{tab:top_mass_systematic_samples}
\end{table}

\subsection{Pile-up reweighting}
\label{ss_top_mass:pileup_reweithing}

\subsection{b-tagging corrections}
\label{ss_top_mass:btagging_corrections}

\section{Event Selection}
\label{s_top_mass:event_selection}


\section{Kinematic Fit}
\label{s_top_mass:kinematic_fit}

\section{Ideogram Method}
\label{s_top_mass:ideogram_method}

\subsection{Event likelihood}
\label{ss_top_mass:event_likelihood}

\subsection{Combined likelihood fit}
\label{ss_top_mass:likelihood_fit}

\section{Mass calibration}
\label{s_top_mass:calibration}

\section{Systematic Uncertainties}
\label{s_top_mass:systematics}

\section{Results}
\label{s_top_mass:results}

\section{Summary}
\label{s_top_mass:summary}


% ------------------------------------------------------------------------


%%% Local Variables: 
%%% mode: latex
%%% TeX-master: "../thesis"
%%% End: 
