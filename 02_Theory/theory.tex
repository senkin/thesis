%!TEX root = ../thesis.tex

\chapter{Theoretical Background}
\label{c:theory}
%or phenomenology?

\ifpdf
    \graphicspath{{02_Theory/plots/}}
\else
    \graphicspath{{02_Theory/plots/EPS/}{02_Theory/plots/}}
\fi

\section{The Standard Model of Particle Physics}
\label{s:SM}
The Standard Model (SM) is a quantum field theory which is currently the most accurate description of matter and all
known fundamental interactions, with the exception of gravity. It was developed in the 20th century as a combination of
two complementary field theories, the Glashow-Weinberg-Salam (GWS) \autocite{Glashow, Weinberg, Salam} theory of
electroweak interaction, and Quantum Chromodynamics (QCD) theory of strong interaction.

The fundamental particles in the Standard Model are the three generations of fermions, shown in
Table~\ref{tab:SM_fermions}, interacting via gauge bosons of three fundamental interactions, presented in
Table~\ref{tab:SM_forces}. Fermions, subdivided into quarks and leptons, are the building blocks of matter in the
observable universe. Their interaction via exchanging the gauge bosons allows formation of hadrons and atoms. The
difference between quarks and leptons mainly lies in the way they interact: quarks can interact via strong,
electromagnetic and weak forces, whereas leptons are only subject to electromagnetic and weak interactions. Each of
these particles also has a corresponding antiparticle with the same mass, but opposite charge.

There are six flavours of leptons in the SM, making up three generations: electron ($e$), muon ($\mu$) and tau ($\tau$)
leptons with their corresponding neutrinos. Electrons, muons and taus are massive and charged particles, interacting via
electromagnetic and weak forces, described in Section~\ref{ss:electroweak_theory}. Neutrinos, on the other hand, do not
carry an electric charge, and therefore are only subject to the weak force, which makes them extremely difficult to
detect as they barely interact with matter. In the classic version of Standard Model neutrinos are considered to be
massless, however, the observation of neutrino oscillations between different flavours \autocite{neutrino_oscillations}
implies that neutrinos are in fact massive. This is one of the deficiencies of the Standard Model, giving rise to
possible extensions elaborated in Section~\ref{s:BSM}.

The quarks also exist in three generations, and are represented by up and down (\cPqu, \cPqd), charm and strange (\cPqc,
\cPqs), and bottom and top (\cPqb, \cPqt) flavours. Together with the mediators of the strong interaction (gluons),
these particles possess another physical property called colour charge. Due to a phenomenon of colour confinement, they
can only exist as constituents of composite particles -- hadrons. More detail on the QCD theory of strong interaction is
presented in Section~\ref{ss:QCD_theory}. The top quark, however, is too heavy hadronise as it has an extremely short
lifetime of \SI{\approx 5e-25}{\s} \autocite{PDG}, which is an order of magnitude shorter than the timescale of strong
interactions. This and other properties of the top quark will be discussed in Section~\ref{s:top_quak_physics}.

The last essential constituent of the Standard Model is the recently discovered Higgs boson
\autocite{ATLAS_higgs_observation, CMS_higgs_observation}. The Higgs field is responsible for electroweak symmetry
breaking of $SU(2)_L \times U(1)_Y$ group and therefore acquisition of mass by vector bosons. This mechanism is
described in more detail in Section~\ref{ss:electroweak_symmetry_breaking}.

\begin{table}[htbp]
\centering
\caption[Fundamental fermions of the Standard Model]{Three generations of fundamental fermions of the Standard Model
with charge and mass properties quoted from the 2012 edition of Review of Particle Physics with 2013 partial update for
the 2014 edition \autocite{PDG, world_top_mass_combination}.}
\label{tab:SM_fermions}
\resizebox{\textwidth}{!} {
\begin{tabular}{@{}_c|^c|^c|^c|^c|^c|^c@{}}
 \toprule
 \multirow{2}{*}[-2pt]{Generation} & \multicolumn{3}{c|}{Leptons} & \multicolumn{3}{c}{Quarks} \\
 \cmidrule{2-7}
  & Flavour & Charge & Mass [\si{\MeV}] & Flavour & Charge & Mass [\si{\MeV}] \\
 \midrule
 \multirow{2}{*}{1} & $e$ & \num{-1} & \num{0.511} & \cPqu & \num{+2/3} & $2.3^{+0.7}_{-0.5}$ \\
                    & $\nu_e$ & \num{0} & $<\num{2e-6}$ & \cPqd & \num{-1/3} & $4.8^{+0.5}_{-0.3}$ \\
 \midrule
 \multirow{2}{*}{2} & $\mu$ & \num{-1} & \num{105.66} & \cPqc & \num{+2/3} & $(1.29^{+0.05}_{-0.11}) \times 10^3$ \\
                    & $\nu_\mu$ & \num{0} & $<\num{0.19}$ & \cPqs & \num{-1/3} & $95 \pm 5$ \\
 \midrule
 \multirow{2}{*}{3} & $\tau$ & \num{-1} & $1776.82 \pm 0.16$ & \cPqt & \num{+2/3} & $(173.34 \pm 0.76) \times 10^3$ \\
                    & $\nu_\tau$ & \num{0} & $<\num{18.2}$ & \cPqb & \num{-1/3} & $(4.18 \pm 0.03) \times 10^3$ \\

%http://pdg.lbl.gov/2013/tables/rpp2013-sum-quarks.pdf

\bottomrule
\end{tabular}}
\end{table}

\begin{table}[thbp]
\centering
\caption[Fundamental forces and corresponding gauge bosons with their properties]{Fundamental forces and corresponding
gauge bosons with their properties \autocite{PDG}. The gravitation is the only fundamental interaction not desribed by
the Standard Model.}
\label{tab:SM_forces} 
\begin{tabular}{@{}_l|^c|^c|^c|^c|^c@{}}
 \toprule
 Force & Gauge Boson & Charge & Spin & Mass \si{[\GeV]} & Range\\ 
 \midrule
 electromagnetic  & photon ($\gamma$) & 0 & 1 & 0 & $\infty$\\
 \midrule
 \multirow{2}{*}{weak} & $\W^\pm$ & \num{\pm1} & \multirow{2}{*}{\num{1}} & $80.385 \pm 0.015$ & \multirow{2}{*}{\SI{d-18}{\metre}}\\   
                       & $\Z^0$   & \num{0} &                             & $91.1876 \pm 0.0021$ &                                 \\
 \midrule
 strong  & 8 gluons (\cPg) & 0 & 1 & 0 & \SI{d-15}{\metre} \\
 \midrule
 gravitational  & graviton (\cPG) & 0 & 2 & 0 & $\infty$\\
\bottomrule
\end{tabular}
\end{table}

\newpage
\subsection{Gauge Principle}
\label{ss:gauge_principle}
As it was mentioned, the Standard Model is a unification of two gauge theories describing electroweak and strong
interactions. The GWS model, a unified theory of electromagnetic and weak interactions, is based on a gauge group of
$SU(2)_L \times U(1)_Y$, whereas the QCD theory has a gauge symmetry of $SU(3)_C$. Therefore, the Standard Model is also
a gauge theory with the symmetry group of $SU(3)_C \times SU(2)_L \times U(1)_Y$. A gauge theory is a theory that is
invariant under a set of local transformations, i.e.\ transformations with space-time dependent parameters. To explain
this concept, let us consider a Lagrangian density of a free Dirac field $\psi$, describing free-moving fermions:
\begin{equation}
\calL = \bar{\psi}(i \gamma^\mu	\partial_\mu - m) \psi
\label{eq:dirac_field_L}
\end{equation}

Clearly, it is invariant under the following phase transformation:
\begin{equation}
\psi \rightarrow \psi' = e^{i \theta} \psi, ~\bar{\psi} \rightarrow \bar{\psi} = e^{-i \theta} \bar{\psi},
\end{equation}
since in the combination $\bar{\psi}\psi$ the exponential factors cancel out ($e^{i \theta} e^{-i \theta} = 1$). This
transformation is called a \textit{global} phase transformation, since the phase $i \theta$ is space-time independent.
The \textit{local} phase transformation, on the other hand, corresponds to the phase $i \theta(x_\mu)$ which is
different at each space-time point:
\begin{equation}
\psi \rightarrow \psi' = e^{i \theta(x)} \psi, ~\bar{\psi} \rightarrow \bar{\psi} = e^{-i \theta(x)} \bar{\psi}.
\end{equation}
Here the space-time indices are dropped on $x_\mu \equiv x$ for clarity. One can see that the Lagrangian is no longer
invariant under such transformation, since
\begin{equation}
\partial_\mu (e^{i \theta(x)} \psi) = i (\partial_\mu \theta(x))  e^{i \theta(x)} \psi + e^{i \theta(x)} \partial_\mu \psi,
\end{equation}
and therefore
\begin{equation}
\calL \rightarrow \calL - [\partial_\mu \theta(x)] \bar{\psi} \gamma^\mu \psi.
\end{equation}

The space-time dependent local transformations are called gauge transformations. In order to restore gauge invariance
under such transformations, we need to substitute the ordinary derivative $\partial_\mu$ with the \textit{covariant}
derivative $D_\mu$:
\begin{equation}
\partial_\mu \rightarrow D_\mu = \partial_\mu + i q A_\mu,
\end{equation}
where $A_\mu$ is a new vector field which transforms under a local gauge transformation as
\begin{equation}
A_\mu \rightarrow A'_\mu = A_\mu - \frac{1}{q} \partial_\mu \theta(x).
\end{equation}

Together with the kinetic energy term of the field-strength tensor defined as
\begin{equation}
\label{eq:F_mu_nu}
F_{\mu\nu} \equiv [\partial_\mu A_\nu] - [\partial_\nu A_\mu],
\end{equation}
we obtain the full gauge invariant QED Lagrangian:
\begin{equation}
\calL_\textrm{QED} = \bar{\psi}(i \gamma^\mu (\partial^\mu + i q A_\mu) - m) \psi - \frac{1}{4}F_{\mu\nu} F^{\mu\nu}.
%\bar{\psi}(i \gamma^\mu \partial^\mu -m) \psi - q \bar{\psi}\gamma^\mu A_\mu \psi - \frac{1}{4}F_{\mu\nu} F^{\mu\nu}.
\end{equation}

The requirement of local gauge invariance necessitates the gauge field $A_\mu$ to be massless, since the term
proportional to $A_\mu A^\mu$ is not invariant under local gauge transformations and therefore has to be excluded from
the Lagrangian. In case of QED, the gauge field $A_\mu$ represents the photon field, and the Lagrangian describes the
interaction of Dirac fields (electrons and positrons) with Maxwell fields (photons).

The idea of requiring a local gauge invariance by introducing additional fields in the Lagrangian to make it covariant
with respect to an extended group of local transformations is an important procedure in particle physics, referred to as
gauge principle. In the example above, the local phase transformation can be thought of as multiplication of the Dirac
field $\psi$ by a $1 \times 1$ unitary matrix ($\psi \rightarrow U \psi$, where $U^{\dag}U = 1$). Here $U = e^{i
\theta(x)}$, and all such transformations form an Abelian Lie group of $U(1)$, since any two its elements commute:
\begin{equation}
[e^{i \theta(x)},~e^{i\theta(x')}] = 0.
\end{equation}

The same strategy of requiring the global invariance to hold locally can be generalised to any $SU(N)$ group: in the
1950s Yang and Mills produced a theory of $SU(2)$ gauge fields, and later on the idea was extended to $SU(3)$ group
giving rise to Quantum Chromodynamics. In fact, all of the fundamental interactions in the Standard Model are introduced
using the same gauge principle.

\newpage
\subsection{Electroweak theory}
\label{ss:electroweak_theory}
In 1961, a partially unified theory combining the electromagnetic and weak interactions was proposed by Sheldon Glashow
\autocite{Glashow}. A few years later the model was independently revised by Weinberg \autocite{Weinberg} and Salam
\autocite{Salam} to introduce massive vector bosons, acquiring mass through spontaneous symmetry breaking. The resulting
electroweak theory, usually referred to as GWS model, is based on a non-Abelian (i.e.\ not commuting) symmetry group of
$SU(2)_L \times U(1)_Y$. Here $U(1)_Y$ is no longer a QED gauge group mentioned previously, but the group associated
with the weak hypercharge $Y$, which relates to the electric charge $Q$ and the third component of weak isospin $I_3$
as
\begin{equation}
Q = I_3 + \frac{Y}{2} 
\end{equation} 

This equation is known as Gell-Mann-Nishijima relation, and the coefficient in front of the weak hypercharge is purely
conventional. The weak force is the only interaction that violates parity, meaning that it can distinguish between
left-handed and right-handed systems -- this fact was famously confirmed by Madame Wu's experiment in 1957
\autocite{Madame_Wu}. To accommodate for this observation, fermion fields are divided into left-handed and right-handed
components as follows:
\begin{equation}
\twovector{\nu_{l,L}}{l_L}, l_R
\label{eq:lepton_EWK_fields}
\end{equation} 
for leptons ($l = e, \mu, \tau$), and
\begin{equation}
\twovector{u_L}{d_L}, u_R, d_R; ~~ \twovector{c_L}{s_L}, c_R, s_R; ~~ \twovector{t_L}{b_L}, t_R, b_R
\label{eq:quark_EWK_fields}
\end{equation} 
for all three generations of quarks. The left-handed particles are doublets with the weak isospin of $I = \frac{1}{2}$,
whereas the right-handed particles are singlets with $I = 0$. These components are obtained by applying the projection
operators $\frac{1}{2}(1\pm\gamma^5)$ to fermionic fields:
\begin{equation}
\psi = \frac{1}{2}(1-\gamma^5) \psi + \frac{1}{2}(1 + \gamma^5)\psi = \psi_L +  \psi_R,
\end{equation} 
where
\begin{equation}
\gamma^5 = i \gamma^0  \gamma^1  \gamma^2  \gamma^3 = \begin{pmatrix}
0 & 1\\
1 & 0
\end{pmatrix}.
\end{equation}

By requiring a local gauge invariance to the group $SU(2)_L \times U(1)_Y$, a covariant derivative is written as
\begin{equation}
\partial_\mu \rightarrow D_\mu = \partial_\mu + i g \mathbf{T} \cdot \mathbf{W}_\mu + i \frac{g'}{2} Y B_\mu.
\label{eq:D_mu_EWK}
\end{equation}

Here $g$ and $g'$ are the coupling constants for groups $SU(2)_L$ and $U(1)_Y$, respectively; $\mathbf{T} =
\hat{\sigma}/2$ are the Pauli matrices which are the generators for $SU(2)$ group; $\mathbf{W}_\mu$ is the triplet of
new gauge fields $W^{1,2,3}_\mu$ introduced for $SU(2)_L$ invariance; $B_\mu$ is the new gauge field for $U(1)_Y$
invariance. For right-handed particles (singlets), $\mathbf{T} = 0$ and the second term in the covariant derivative
vanishes. The resulting gauge-invariant Lagrangian for electroweak theory with kinetic energy terms takes form:
\begin{equation}
\label{eq:EWK_L}
\calL_\textrm{EWK} = \bar{\psi}_L \gamma^\mu ( i\partial_\mu  - g \mathbf{T} \cdot \mathbf{W}_\mu - \frac{g'}{2} Y
B_\mu) \psi_L + \bar{\psi}_R \gamma^\mu (i \partial_\mu - \frac{g'}{2} Y B_\mu) \psi_R - \frac{1}{4}\mathbf{W}_{\mu\nu}
\mathbf{W}^{\mu\nu} -\frac{1}{4}B_{\mu\nu} B^{\mu\nu},
\end{equation}
where
\begin{equation*}
\psi_L = \twovector{\psi^1_L}{\psi^2_L},
\end{equation*}
and $\psi_L$ and $\psi_R$ are summed over all possibilities shown in Equations~\ref{eq:lepton_EWK_fields} and
\ref{eq:quark_EWK_fields}.

The newly introduced gauge fields $W^{1,2,3}_\mu$ and $B_\mu$ do not directly correspond to the physical fields of
gauge bosons, but form them via the following linear combinations:
\begin{subequations}
\begin{align}
W^{\pm}_\mu & = \frac{1}{\sqrt{2}} (W^1_\mu \pm  i W^2_\mu), \label{eq:W_mu} \\
Z_\mu & = - B_\mu \sin{\theta_W} + W^3_\mu \cos{\theta_W}, \label{eq:Z_mu} \\
A_\mu & = B_\mu \cos{\theta_W} + W^3_\mu \sin{\theta_W}. \label{eq:A_mu}
\end{align}
\end{subequations}

We obtained the physical fields of charged $W^{\pm}$ bosons, as well as a neutral $Z^0$ boson and a photon, which were
produced by mixing the neutral fields $W^3_\mu$ and $B_\mu$ with a rotation by the weak mixing (or
\textit{Weinberg}) angle $\theta_W$. The electric charge is then given as
\begin{equation}
e = g' \cos{\theta_W} = g \sin{\theta_W}.
\end{equation}

Up to this point, there are no mass terms in the Lagrangian as their introduction would violate the gauge invariance.
However, the existence of massive \W and \Z bosons clearly contradicts the notion of massless gauge fields in
electroweak theory. This problem is solved by the Higgs mechanism of spontaneous gauge symmetry breaking, which is a
subject of Section~\ref{ss:electroweak_symmetry_breaking}.

\newpage
\subsection{Quantum Chromodynamics}
\label{ss:QCD_theory}
Quantum Chromodynamics is the quantum field theory of strong interaction between quarks and gluons. It is a non-Abelian
gauge theory based on the symmetry group $SU(3)_C$, where $C$ denotes colour charge. This is motivated mainly by
experimental evidence, which suggests that quarks are multiplets of fields in colour space:

\begin{equation}
q = \threevector{r}{g}{b}.
\end{equation} 

After demanding the invariance
under local $SU(3)$ gauge transformations, the QCD Lagrangian is given as
\begin{equation}
\label{eq:QCD_L}
\calL_\textrm{QCD} = \bar{q} (\gamma^\mu \partial_\mu -m)q + g_S (\bar{q} \gamma^\mu T_a q) G^a_\mu -  \frac{1}{4} G^a_{\mu\nu}
G_a^{\mu\nu}.
\end{equation}

Here $T_a = \lambda_a / 2$ are the Gell-Mann matrices, i.e.\ eight generators of the $SU(3)$ group, satisfying the Lie
algebra:
\begin{equation}
[T_a,T_b] = i f_{abc} T_c.
\end{equation}

This gives rise to eight gauge fields -- massless gluons. The field strength $G^a_{\mu\nu}$ is tensor is analogous to
that of QED (Equation~\ref{eq:F_mu_nu}), but contains an extra term proportional to the structure constants $f^{abc}$,
responsible for self-interaction of gluons:
\begin{equation}
\label{eq:G_mu_nu}
G^a_{\mu\nu} = [\partial_\mu G^a_\nu] - [\partial_\nu G^a_\mu] + g_S f^{abc} G^b_\mu G^c_\nu.
\end{equation}

Notably, leptons are singlets under $SU(3)$ symmetry, and therefore do not participate in strong interaction with gluons
and quarks. Self-interaction of gluons leads to two distinctive fundamental properties of QCD: asymptotic freedom and
colour confinement.

\begin{description}[wide=\parindent]
\item [Asymptotic freedom] was firstly described by Frank Wilczek, David Gross \autocite{Gross_Wilczek}, and
independently by David Politzer \autocite{Politzer} in 1973. This phenomenon is closely connected to the behaviour of
QCD gauge coupling $g_S$, which is usually presented in terms of $\alpha_S$ defined as
\begin{equation}
\alpha_S (\mu) = \frac{g_S (\mu)^2}{4 \pi},
\end{equation}
where $\mu$ is the energy scale of the process. It was discovered that $\alpha_S$ \textit{decreases} with the increase
of energy scale. This can be shown by evaluating contributions from quark-antiquark and (crucially) gluon-gluon loops,
leading to the first order to \autocite{Griffiths}
\begin{equation}
\alpha_S (\mu) = \frac{12\pi}{(33-2n_f) \ln(\mu^2/\Lambda_{\text{QCD}}^2)},
\label{eq:alpha_S}
\end{equation}
where $n_f$ is the number of flavours available at the energy scale $\mu$, and $\Lambda_{QCD}$ is the QCD scale, i.e.\
the value of renormalisation scale at which the QCD coupling diverges. Since the number of quark flavours is known to be
less than 16, the coefficient by the logarithm in Equation~\ref{eq:alpha_S} is positive, meaning that $\alpha_S$ in fact
decreases with the increase of $\mu$ (or decrease of the probe distance). An important consequence of this phenomenon is
the fact that at very short distances (\SI{\sim0.1}{\femto\metre}), the strong force is relatively weak and quarks
become effectively free (hence the name ``asymptotic freedom'').

\item [Colour confinement] is another important consequence -- since the strong force increases with distance between
quarks, in case if the quarks move apart the energy rises until being sufficient to form a quark-antiquark pair. This
process is called hadronisation, and it continues until the kinetic energy of the quarks is completely transformed into
the energy of creation of new quark pairs. All quarks are therefore confined in colourless bound states, called hadrons.
Hadrons formed by a quark-antiquark pair are called mesons, whereas combinations of three quarks or three antiquarks are
referred to as baryons. Bound states with higher number of quarks (e.g.\ pentaquarks), or containing gluons only
(glueballs) remain unobserved, despite being predicted by the Standard Model.

\end{description}

In Section~\ref{ss:electroweak_theory} we introduced weak isospin doublets for three generations of left-handed quarks
(Equation~\ref{eq:quark_EWK_fields}). These states, however, do not exactly correspond to the strong force eigenstates.
Experimental evidence (e.g.\ $n \rightarrow p e^- \bar{\nu_e}$ decay) suggests that the weak interaction is capable of
changing the quark flavour, meaning that the weak interaction eigenstates are in fact mixtures of flavour eigenstates.
Relation between the two sets of eigenstates is given by the Cabbibo-Kobayashi-Maskawa (CKM) matrix:
\begin{equation}
\threevector{d'}{s'}{b'}
=
\begin{pmatrix}
V_{ud} & V_{us} & V_{ub}  \\
V_{cd} & V_{cs} & V_{cb}  \\
V_{td} & V_{ts} & V_{tb}
\end{pmatrix}
\threevector{d}{s}{b},
\label{eq:CKM}
\end{equation}
where $d'$, $s'$, $b'$ are the weak interaction eigenstates, and $d$, $s$, $b$ are the strong force eigenstates. The CKM
matrix describes flavour mixing, i.e.\ the probability of transitions between different quark flavours through weak
interactions. The origin of such mixing lies in the Yukawa interaction between quarks and the Higgs field, described in
the next section.

% The magnitudes of the CKM matrix elements are\autocite{PDG}:
% \begin{equation}
% \begin{pmatrix}
% \abs{V_{ud}} & \abs{V_{us}} & \abs{V_{ub}}  \\
% \abs{V_{cd}} & \abs{V_{cs}} & \abs{V_{cb}}  \\
% \abs{V_{td}} & \abs{V_{ts}} & \abs{V_{tb}}
% \end{pmatrix} =
% \begin{pmatrix}
% 0.97428 \pm 0.00015 & 0.2253 \pm 0.0007 &  0.00347^{+0.00016}_{-0.00012} \\
% 0.2252 \pm 0.0007 & 0.97345^{+0.00015}_{-0.00016} & 0.0410^{+0.0011}_{-0.0007}  \\
% 0.00862^{+0.00026}_{-0.00020} & 0.0403^{+0.0011}_{-0.0007} & 0.999152^{+0.000030}_{-0.000045}
% \end{pmatrix}.
% \end{equation}

\subsection{Electroweak symmetry breaking}
\label{ss:electroweak_symmetry_breaking}
The mechanism of spontaneous electroweak symmetry breaking, often called the Higgs mechanism, was developed in the 1960s
by Peter Higgs \autocite{Higgs} and two other groups independently: Robert Brout and Fran\c{c}ois Englert
\autocite{Englert_Brout}; Gerald Guralnik, Carl Richard Hagen, and Tom Kibble \autocite{Guralnik_Hagen_Kibble}. As it
was already mentioned, this mechanism is responsible for acquisition of mass by the vector bosons, since direct
inclusion of mass terms in the electroweak Lagrangian (Equation~\ref{eq:EWK_L}) would violate gauge symmetry. Therefore,
we need to spontaneously break the $SU(2)$ symmetry by adding an external field with a non-zero vacuum expectation
value. This is done by introducing an $SU(2)$ doublet of complex scalar fields, called the Higgs fields:
\begin{equation}
\phi = \twovector{\phi^+}{\phi^0} = \frac{1}{\sqrt{2}} \twovector{\phi_1 + i \phi_2}{\phi_3 + i \phi_4},
\end{equation}
with the following additional term in the Lagrangian:
\begin{equation}
\calL_\textrm{Higgs} = (D_\mu \phi)^{\dag} (D^\mu \phi) - V(\phi),
\label{eq:Higgs_L}
\end{equation}
where $D_\mu$ is the electroweak covariant derivative from Equation~\ref{eq:D_mu_EWK}. The scalar potential $V(\phi)$
is given by
\begin{equation}
V(\phi) = - \mu^2(\phi^{\dag} \phi) + \lambda(\phi^{\dag} \phi)^2.
\end{equation}

With the choice of $\mu^2>0$ and $\lambda>0$, the potential has the shape shown in Figure~\ref{fig:higgs_potential}.
Clearly, the minimum of this potential is not at $\phi = 0$, but forms a circle in $SU(2)$ space specified by
\begin{equation}
(\phi^{\dag} \phi)_\textrm{min} = \frac{\mu^2}{2\lambda} = \frac{v^2}{2}.
\end{equation}

\begin{figure}[!hbtp]
\centering
\includegraphics[width=0.5\textwidth]{Higgs_potential}
\caption[The Higgs field potential.]{The Higgs field potential in the complex plane, commonly referred to as a
``Mexican hat'' potential.}
\label{fig:higgs_potential}
\end{figure}

The choice of minimum corresponding to the lowest energy state (or vacuum) is completely arbitrary. In fact, any point
at which the potential is minimum loses the invariance under $SU(2)_L \times U(1)_Y$ gauge transformations. Therefore,
nature spontaneously breaks the symmetry by picking the vacuum from the set of minima of the Higgs potential.
Conventionally, we can choose
\begin{equation}
\langle 0 | \phi | 0 \rangle = \frac{1}{\sqrt{2}} \twovector{0}{v}.
\label{eq:unitary_gauge_vev}
\end{equation}

Expanding around the chosen minimum, $\phi$ is then given by
\begin{equation}
\phi = \frac{1}{\sqrt{2}} \twovector{0}{v+H},
\end{equation}
where $H$ is the neutral scalar Higgs field. By substituting this field into the Lagrangian in
Equation~\ref{eq:Higgs_L}, one can obtain:
\begin{equation}
\begin{split}
\calL_\textrm{Higgs} & = \frac{1}{2} (\partial_\mu H)(\partial^\mu H) + \frac{1}{4} (H^2 + 2vH + v^2) W^+_\mu W^{-\mu} + \\
& + \frac{1}{8} (g^2 + {g'}^2)(H^2 + 2vH + v^2)Z_\mu Z^\mu - \\
& - \mu^2 H^2 - \frac{\lambda}{4} (H^4 + 4vH^3),
\end{split}
\label{eq:Higgs_L_final}
\end{equation}
where physical fields $W^{\pm}_\mu$ and $Z_\mu$ are given by Equations~\ref{eq:W_mu} and \ref{eq:Z_mu}, respectively.
Notably, we can see that these fields have acquired mass terms in the Lagrangian, with the masses of vector bosons given
by
\begin{subequations}
\begin{align}
M_W &= \frac{1}{2}gv, \\
M_Z &= \frac{1}{2}v\sqrt{g^2+{g'}^2} = \frac{1}{2} \frac{gv}{\cos{\theta_W}},
\end{align}
\end{subequations}
which was confirmed by experimental measurements. Naturally, the photon field $A_\mu$ acquires no mass terms in the
Lagrangian. The Higgs mechanism can also be used in a similar way to generate fermion masses by introducing an $SU(2)_L
\times U(1)_Y$ gauge invariant term responsible for interaction between the Higgs and fermion fields. This additional
term in the Standard Model Lagrangian is called \textit{Yukawa term}, which for the first generation of fermions is
given by
\begin{equation}
\calL_\textrm{Yukawa} = -Y_e^{ij} \bar{l_L}^i \phi e_R^j - Y_u^{ij} \bar{q_L}^i \epsilon \phi^\dag u_R^j - Y_d^{ij}
\bar{q_L}^i \phi d_R^j + \textrm{h.c.},
\end{equation}
where coefficients $Y_{e,u,d}^{ij}$ are $3\times3$ complex matrices (Yukawa couplings) and $\epsilon$ is the $2\times2$
antisymmetric tensor. When the Higgs field $\phi$ acquires the vacuum expectation value given by
Equation~\ref{eq:unitary_gauge_vev}, the Yukawa Lagrangian yields mass terms for fermions, generating the masses:
\begin{equation}
M_f = Y_f \frac{v}{\sqrt{2}}.
\end{equation}

The Yukawa couplings are not diagonal in general, which results in mixing between different generations described for
the quarks by the CKM matrix (Equation~\ref{eq:CKM}). Another curious observation is that the fermion masses are
proportional to Yukawa couplings, which essentially represent the interaction strength with the Higgs field. Due to the
large mass of the top quark of approximately \SI{173}{\GeV}, and the Higgs field vacuum expectation value
$v\approx\SI{246}{\GeV}$, the Yukawa coupling to the top quark is very close to unity.

Returning to the Lagrangian in Equation~\ref{eq:Higgs_L_final}, the Higgs boson mass is
\begin{equation}
M_H = \sqrt{2} \mu = v \sqrt{2\lambda}.
\end{equation}

Unlike the \W and \Z boson masses which can be predicted from the measurement of the fine structure constant, the Higgs
mass $M_H$ can not be determined by other experimentally measured parameters. However, it is possible to put indirect
constraints on $M_H$ through quantum loop corrections and precise measurement of the \W boson and top quark masses. This
will be discussed in Section~\ref{s:top_quak_physics}.

The existence of the Higgs boson and therefore the nature of electroweak symmetry breaking had long remained a mystery.
In the summer of 2012 both ATLAS and CMS experiments at the LHC observed the Higgs boson with an approximate mass of
\SI{125}{\GeV} \autocite{ATLAS_higgs_observation, CMS_higgs_observation}, with couplings consistent with theoretical
predictions, which proved to be yet another triumph of the Standard Model.

\newpage
\section{Top Quark Physics within the Standard Model}
\label{s:top_quak_physics}
The top quark is the heaviest elementary particle in the Standard Model, and the heaviest observed particle, discovered
in 1995 at the Tevatron collider by CDF and D{\O} collaborations \autocite{CDF_top_observation, D0_top_observation}. The
large mass of $\mtop = 173.29\ \pm\ 0.23\ \textrm{(stat.)}\ \pm 0.92\ \textrm{(syst.)}\ \textrm{GeV}$
\autocite{LHC_top_mass_combination} suggests that the top quark may have a special role in nature. It is substantially
more massive than any other fermion. The Standard Model does not provide an explanation for the vast difference between
masses of fermions (known as the Flavour problem), hence the origin of Yukawa couplings and therefore the CKM matrix
remains an open problem in particle physics. It is not yet understood why the top quark's Yukawa coupling to the Higgs
boson is so close to unity. For these reasons, the top quark plays an important role in many models beyond the Standard
Model, discussed in Section~\ref{s:BSM}.

Due to its large mass, the top quark has an extremely short lifetime of \SI{\approx5e-25}{\s} \autocite{PDG}, meaning
that it decays before the top-flavoured hadrons or \ttbar bound states can form. Therefore, the spin information is
passed from the top quark on to its decay products, providing a unique possibility to study a ``bare'' quark.
High-precision measurements of the top quark properties, such as mass (Chapter~\ref{c:top_mass_analysis}) and cross
section (Chapter~\ref{c:xsection_analysis}) are important tools to test the Standard Model and probe for new physics
beyond it.

\subsection{Top quark production at the LHC}
\label{ss:top_production}
There are two particular mechanisms of top quark production in hadron collisions: top-antitop (\ttbar) pair production
via strong interaction, and single top quark production via electroweak interaction. The \ttbar production dominates
over single top production, being the main source of top quarks at the LHC, and therefore is the main focus of this
thesis.

Feynman diagrams of \ttbar production to the leading order are presented in
Figure~\ref{fig:ttbar_production_feynman_diagrams}. At the LHC, the major process for \ttbar production is gluon fusion,
constituting approximately \SI{90}{\pc} (\SI{80}{\pc}) at $\sqrt s =$ \SI{14}{\TeV} (\SI{7}{\TeV}) \autocite{PDG}.
Quark-antiquark annihilation and higher order processes contribute to the rest of the Standard Model \ttbar production.
This happens due to the fact that the LHC is a proton-proton collider, and the only source of antiquarks are virtual
quarks (so-called sea quarks). At the Tevatron, which was a proton-antiproton collider, the situation was quite the
opposite: the aquark-antiquark annihilation was the main \ttbar production mode.

\begin{figure}[!htbp]
	\begin{minipage}[t]{0.49\textwidth}
	\centering
	\subfloat[]{
		\begin{fmfgraph*}(150,75)
		\fmfleft{i1,i2}
		\fmfright{o1,o2}
		\fmflabel{$g$}{i1}
		\fmflabel{$g$}{i2}
		\fmflabel{$\bar{t}$}{o1}
		\fmflabel{$t$}{o2}

		\fmf{gluon}{i1,v1}
		\fmf{gluon}{i2,v2}
		\fmf{fermion}{v1,v2}
		\fmf{fermion}{o1,v1}
		\fmf{fermion}{v2,o2}
		\end{fmfgraph*}
	}
	\end{minipage}
	\hfill
	\begin{minipage}[t]{0.49\textwidth}
	\centering
	\subfloat[]{
		\begin{fmfgraph*}(150,75)
		\fmfleft{i1,i2}
		\fmfright{o1,o2}
		\fmflabel{$g$}{i1}
		\fmflabel{$g$}{i2}
		\fmflabel{$\bar{t}$}{o1}
		\fmflabel{$t$}{o2}

		\fmf{gluon}{i1,v1}
		\fmf{gluon}{i2,v1}
		\fmf{gluon}{v1,v2}
		\fmf{fermion}{o1,v2}
		\fmf{fermion}{v2,o2}
		\end{fmfgraph*}
	}
	\end{minipage}

	\vspace{1cm}

	\begin{minipage}[t]{0.49\textwidth}
	\centering
	\subfloat[]{
		\begin{fmfgraph*}(150,75)
		\fmfleft{i1,i2}
		\fmfright{o1,o2}
		\fmflabel{$g$}{i1}
		\fmflabel{$g$}{i2}
		\fmflabel{$\bar{t}$}{o1}
		\fmflabel{$t$}{o2}

		\fmf{gluon}{i1,v1}
		\fmf{phantom}{v1,o1}
		\fmf{gluon}{i2,v2}
		\fmf{phantom}{v2,o2}
		\fmf{fermion}{v2,v1}

		\fmf{fermion,tension=0}{v1,o2}
		\fmf{fermion,tension=0}{o1,v2}

		\end{fmfgraph*}
	}
	\end{minipage}
	\hfill
	\begin{minipage}[t]{0.49\textwidth}
	\centering
	\subfloat[]{
		\begin{fmfgraph*}(150,75)
		\fmfleft{i1,i2}
		\fmfright{o1,o2}
		\fmflabel{$\bar{q}$}{i1}
		\fmflabel{$q$}{i2}
		\fmflabel{$\bar{t}$}{o1}
		\fmflabel{$t$}{o2}

		\fmf{fermion}{v1,i1}
		\fmf{fermion}{i2,v1}
		\fmf{gluon}{v1,v2}
		\fmf{fermion}{o1,v2}
		\fmf{fermion}{v2,o2}
		\end{fmfgraph*}
	}
	\end{minipage}

	\caption[Feynman diagrams for leading order \ttbar production at the LHC.]{Feynman diagrams for leading order \ttbar
	production at the LHC. (a), (b) and (c) show the gluon fusion, the dominant production mechanism, (d) represents
	quark-antiquark annihilation.}
  \label{fig:ttbar_production_feynman_diagrams}
\end{figure}

The \ttbar production is a strong interaction process, and therefore is described by perturbative QCD. Hadron collisions
at the LHC (or other hadron colliders) are best viewed as interactions between their constituent quarks and gluons, also
referred to as partons. Since collisions occur at high energies, these interactions result in hard scattering processes
between incoming partons, meaning those involving a significant momentum transfer comparing to the proton mass, and
potentially giving rise to highly energetic final states like top quarks. Each incoming parton carries only a fraction
$x$ of the total momentum of a parent hadron. The distribution of momentum fractions for all flavours of partons are
described by parton distribution functions (PDFs). A PDF $f_i(x_i,\mu_f^2)$ is defined as a probability density of
finding a parton with flavour $i$ and a longitudinal momentum fraction $x_i$ when probed at momentum scale $\mu_f^2$,
which is known as factorisation scale. This parameter separates the hard scattering process into the hard partonic
interaction and the soft (or long-range) interaction. The former interaction happens at a short distance, and therefore
only involves high-momentum transfer which is calculable in perturbative QCD. The soft long-range part of the
interaction, on the contrary, can not be calculated in QCD and instead is parametrised by the PDFs, which have to be
obtained from experimental data. Figure~\ref{fig:CT10_PDFs} shows one of the latest PDFs obtained by CTEQ-TEA
collaboration \autocite{CT10_NNLO}.

\begin{figure}[!hbtp]
   \centering
   \subfloat[]{\includegraphics[width=0.5\textwidth]{CT10_1}}
   \subfloat[]{\includegraphics[width=0.5\textwidth]{CT10_2}}
   \caption[CT10 NNLO parton distribution functions.]{CT10 NNLO parton distribution functions for up and down quarks,
   gluons and sea quarks, given at two different factorisation scales \autocite{CT10_NNLO}.}
   \label{fig:CT10_PDFs}
\end{figure}


The total \ttbar production cross section for hard scattering processes in hadron collisions can be calculated as
\autocite{Sterman1986}:
\begin{equation}
\sigma^{\ttbar} (s,\mtop^2) = \sum_{i,j} \int dx_i dx_j f_i(x_i, \mu_f^2) f_j(x_j,\mu_f^2) \hat{\sigma}_{i,j
\rightarrow \ttbar} (s, \mtop^2, \alpha_s(\mu_r^2)),
\end{equation}
where the indices $i,j$ run over the incoming partons (gluons and quark-antiquark pairs), $x_{i,j}$ are the momentum
fractions of the incoming partons, $f_{i,j}$ are their parton distribution functions, and $\hat{\sigma}_{i,j
\rightarrow \ttbar}$ is the cross section of interacting partons, which depends on the centre of mass energy of hadron
collisions $s$, the top quark mass \mtop, and the QCD strong coupling constant $\alpha_s$. The latter parameter is
evaluated at renormalisation scale $\mu_r^2$. %explain?

The results of most recent theoretical calculation of \ttbar cross section performed with next-to-next-to-leading-order
(NNLO) QCD corrections \autocite{NNLO_ttbar} are given in Table~\ref{tab:ttbar_NNLO_xsections}. The cross section has
also been measured by CMS and ATLAS experiments at the LHC -- the comparison of these measurements with the theoretical
prediction is shown in Figure~\ref{fig:xsections_comparison_NNLO}.

\begin{table}[!hbp]
\centering
\begin{tabular}{lrrr}
 \toprule
 $\sqrt{s}$ & $\sigma_\textrm{total}$ [\pb] & scales [\pb] & PDF [\pb] \\
 \midrule
 \SI{7}{\TeV}  & 172.0 & ${}^{+4.4~(2.6\%)}_{-5.8~(3.4\%)}$ & ${}^{+4.7~(2.7\%)}_{-4.8~(2.8\%)}$ \\
 \addlinespace[0.5em]
 \SI{8}{\TeV}  & 245.8 & ${}^{+6.2~(2.5\%)}_{-8.4~(3.4\%)}$ & ${}^{+6.2~(2.5\%)}_{-6.4~(2.6\%)}$ \\
 \addlinespace[0.5em]
 \SI{14}{\TeV} & 953.6 & ${}^{+22.7~(2.4\%)}_{-33.9~(3.6\%)}$ & ${}^{+16.2~(1.7\%)}_{-17.8~(1.9\%)}$ \\
 \bottomrule
\end{tabular}
\caption{Theoretical predictions for \ttbar production cross section at
different LHC centre of mass energies, calculated at
next-to-next-to-leading-order (NNLO) \autocite{NNLO_ttbar}. The scales uncertainty corresponds to the choice of factorisation and renormalisation scales. }
\label{tab:ttbar_NNLO_xsections} 
\end{table}


\begin{figure}[!hbtp]
   \centering
   {\includegraphics[width=0.7\textwidth]{xsections_comparison_NNLO}}
   \caption[Measurement of the \ttbar production cross section.]{Measurement of the \ttbar production cross section at
   centre of mass energies of \SI{7}{\TeV} and \SI{8}{\TeV} by ATLAS and CMS experiments, compared with the theoretical
   prediction \autocite{NNLO_ttbar}.}
   \label{fig:xsections_comparison_NNLO}
\end{figure}


Figure~\ref{fig:single_top_production_feynman_diagrams} shows the Feynman diagrams of various single top production
modes. Three major production mechanisms include s-channel and t-channel \W boson exchange, and associated production
with a \W boson (tW-channel). Measurement of single top production is of significant interest since it allows the direct
measurement of the Wtb vertex and therefore the magnitude of $|V_{tb}|$ element of the CKM matrix.

\begin{figure}[!hbtp]
	\centering
	\vspace*{0.5cm}
	\begin{minipage}[b]{0.3\textwidth}
	\centering
	\subfloat[]{
		\begin{fmfgraph*}(150,75)
		\fmfleft{i1,i2}
		\fmfright{o1,o2}
		\fmflabel{$q$}{i1}
		\fmflabel{$\bar{q'}$}{i2}
		\fmflabel{$\bar{b}$}{o1}
		\fmflabel{$t$}{o2}

		\fmf{fermion}{i1,v1}
		\fmf{fermion}{v1,i2}
		\fmf{photon, label=$W$}{v1,v2}
		\fmf{fermion}{o1,v2}
		\fmf{fermion}{v2,o2}
		\end{fmfgraph*}
	}
	\end{minipage}
	\hfill
	\begin{minipage}[b]{0.3\textwidth}
	\centering
	\subfloat[]{
		\begin{fmfgraph*}(150,75)
		\fmfleft{i1,i2}
		\fmfright{o1,o2,o3}
		\fmflabel{$g$}{i1}
		\fmflabel{$q$}{i2}
		\fmflabel{$\bar{b}$}{o1}
		\fmflabel{$t$}{o2}
		\fmflabel{$q'$}{o3}

		\fmf{gluon}{i1,v1}
		\fmf{fermion}{i2,v3}
		\fmf{fermion, label=$b$}{v1,v2}
		\fmf{photon, label=$W$}{v2,v3}
		\fmf{fermion}{o1,v1}
		\fmf{fermion}{v2,o2}
		\fmf{fermion}{v3,o3}
		\end{fmfgraph*}
	}
	\end{minipage}
	\hfill
	\begin{minipage}[b]{0.3\textwidth}
	\centering
	\subfloat[]{
		\begin{fmfgraph*}(150,75)
		\fmfleft{i1,i2}
		\fmfright{o1,o2}
		\fmflabel{$b$}{i1}
		\fmflabel{$q$}{i2}
		\fmflabel{$t$}{o1}
		\fmflabel{$q'$}{o2}

		\fmf{fermion}{i1,v1}
		\fmf{fermion}{i2,v2}
		\fmf{photon, label=$W$}{v1,v2}
		\fmf{fermion}{v1,o1}
		\fmf{fermion}{v2,o2}
		\end{fmfgraph*}
	}
	\end{minipage}

	\vspace{1cm}

	\hfill

	\begin{minipage}[t]{0.3\textwidth}
	\centering
	\subfloat[]{
		\begin{fmfgraph*}(150,75)
		\fmfleft{i1,i2}
		\fmfright{o1,o2}
		\fmflabel{$g$}{i1}
		\fmflabel{$b$}{i2}
		\fmflabel{$t$}{o1}
		\fmflabel{$W$}{o2}

		\fmf{gluon}{i1,v1}
		\fmf{fermion}{i2,v1}
		\fmf{fermion, label=$b$}{v1,v2}
		\fmf{fermion}{v2,o1}
		\fmf{photon}{v2,o2}
		\end{fmfgraph*}
	}
	\end{minipage}
	\hspace{2cm}
	\begin{minipage}[t]{0.3\textwidth}
	\centering
	\subfloat[]{
		\begin{fmfgraph*}(150,75)
		\fmfleft{i1,i2}
		\fmfright{o1,o2}
		\fmflabel{$b$}{i1}
		\fmflabel{$g$}{i2}
		\fmflabel{$W$}{o1}
		\fmflabel{$t$}{o2}

		\fmf{fermion}{i1,v1}
		\fmf{gluon}{i2,v2}
		\fmf{fermion, label=$t$}{v1,v2}
		\fmf{photon}{v1,o1}
		\fmf{fermion}{v2,o2}
		\end{fmfgraph*}
	}
	\end{minipage}
	\hfill

	\caption[Feynman diagrams for leading order single top production.]{Feynman diagrams for leading order
	single top production. (a) s-channel, (b) and (c) t-channel, (d) and (e) tW-channel.}
  \label{fig:single_top_production_feynman_diagrams}
\end{figure}




\subsection{Top quark decay}
\label{ss:top_decay}


\begin{figure}[!hbtp]
	\centering
	\begin{fmfgraph*}(300,150)
	\fmfleft{i1,i2,i3}
	\fmfright{o1,o2,o3}
	\fmflabel{$\bar{q}$}{i1}
	\fmflabel{$q$}{i2}
	\fmflabel{$\bar{b}$}{i3}
	\fmflabel{$b$}{o1}
	\fmflabel{$l^+$}{o2}
	\fmflabel{$\nu_l$}{o3}
	
	\fmf{fermion}{i3,v2}
	\fmf{phantom}{i1,v2}
	\fmf{fermion, label=$\bar{t}$}{v2,v3}
	\fmf{fermion, label=$t$}{v3,v4}
	\fmf{fermion}{v4,o1}
	\fmf{phantom}{o3,v4}
	\fmffreeze
	\fmf{fermion}{i1,v1}
	\fmf{fermion}{v1,i2}
	\fmf{fermion}{o2,v5}
	\fmf{fermion}{v5,o3}
	\fmf{photon, label=$W^-$}{v1,v2}
	\fmf{photon, label=$W^+$}{v4,v5}
	\fmfdot{v3}
	\end{fmfgraph*}
	\caption[Semileptonic \ttbar decay mode.]{Semileptonic \ttbar decay mode.}
  \label{fig:ttbar_semileptonic_decay}
\end{figure}

\subsection{Background processes}
\label{ss:backgrounds}

\subsection{Top quark mass}
\label{ss:top_mass}

\subsection{Importance of top quark physics}
\label{ss:importance}

\section{Physics Beyond The Standard Model}
\label{s:BSM}
% There are simple extensions to the Standard Model describing massive neutrinos and keeping the local symmetry of weak
% interactions.

\subsection{Supersymmetry}
\label{ss:SUSY}

\subsection{Extra dimensions}
\label{ss:extra_dimensions}

% \subsection{Standard Model Lagrangian}

% \begin{center}
% \begin{math}
% \calL = -\frac{1}{2}\partial_{\nu}g^{a}_{\mu}\partial_{\nu}g^{a}_{\mu}
% -g_{s}f^{abc}\partial_{\mu}g^{a}_{\nu}g^{b}_{\mu}g^{c}_{\nu}
% -\frac{1}{4}g^{2}_{s}f^{abc}f^{ade}g^{b}_{\mu}g^{c}_{\nu}g^{d}_{\mu}g^{e}_{\nu}
% +\frac{1}{2}ig^{2}_{s}(\bar{q}^{\sigma}_{i}\gamma^{\mu}q^{\sigma}_{j})g^{a}_{\mu}
% +\bar{G}^{a}\partial^{2}G^{a}+g_{s}f^{abc}\partial_{\mu}\bar{G}^{a}G^{b}g^{c}_{\mu}
% -\partial_{\nu}W^{+}_{\mu}\partial_{\nu}W^{-}_{\mu}-M^{2}W^{+}_{\mu}W^{-}_{\mu}
% -\frac{1}{2}\partial_{\nu}Z^{0}_{\mu}\partial_{\nu}Z^{0}_{\mu}-\frac{1}{2c^{2}_{w}}
% M^{2}Z^{0}_{\mu}Z^{0}_{\mu}
% -\frac{1}{2}\partial_{\mu}A_{\nu}\partial_{\mu}A_{\nu}
% -\frac{1}{2}\partial_{\mu}H\partial_{\mu}H-\frac{1}{2}m^{2}_{h}H^{2}
% -\partial_{\mu}\phi^{+}\partial_{\mu}\phi^{-}-M^{2}\phi^{+}\phi^{-}
% -\frac{1}{2}\partial_{\mu}\phi^{0}\partial_{\mu}\phi^{0}-\frac{1}{2c^{2}_{w}}M\phi^{0}\phi^{0}
% -\beta_{h}[\frac{2M^{2}}{g^{2}}+\frac{2M}{g}H+\frac{1}{2}(H^{2}+\phi^{0}\phi^{0}+2\phi^{+}\phi^{-%%@
% })]+\frac{2M^{4}}{g^{2}}\alpha_{h}
% -igc_{w}[\partial_{\nu}Z^{0}_{\mu}(W^{+}_{\mu}W^{-}_{\nu}-W^{+}_{\nu}W^{-}_{\mu})
% -Z^{0}_{\nu}(W^{+}_{\mu}\partial_{\nu}W^{-}_{\mu}-W^{-}_{\mu}\partial_{\nu}W^{+}_{\mu})
% +Z^{0}_{\mu}(W^{+}_{\nu}\partial_{\nu}W^{-}_{\mu}-W^{-}_{\nu}\partial_{\nu}W^{+}_{\mu})]
% -igs_{w}[\partial_{\nu}A_{\mu}(W^{+}_{\mu}W^{-}_{\nu}-W^{+}_{\nu}W^{-}_{\mu})
% -A_{\nu}(W^{+}_{\mu}\partial_{\nu}W^{-}_{\mu}-W^{-}_{\mu}\partial_{\nu}W^{+}_{\mu})
% +A_{\mu}(W^{+}_{\nu}\partial_{\nu}W^{-}_{\mu}-W^{-}_{\nu}\partial_{\nu}W^{+}_{\mu})]
% -\frac{1}{2}g^{2}W^{+}_{\mu}W^{-}_{\mu}W^{+}_{\nu}W^{-}_{\nu}+\frac{1}{2}g^{2}
% W^{+}_{\mu}W^{-}_{\nu}W^{+}_{\mu}W^{-}_{\nu}
% +g^2c^{2}_{w}(Z^{0}_{\mu}W^{+}_{\mu}Z^{0}_{\nu}W^{-}_{\nu}-Z^{0}_{\mu}Z^{0}_{\mu}W^{+}_{\nu}
% W^{-}_{\nu})
% +g^2s^{2}_{w}(A_{\mu}W^{+}_{\mu}A_{\nu}W^{-}_{\nu}-A_{\mu}A_{\mu}W^{+}_{\nu}
% W^{-}_{\nu})
% +g^{2}s_{w}c_{w}[A_{\mu}Z^{0}_{\nu}(W^{+}_{\mu}W^{-}_{\nu}-W^{+}_{\nu}W^{-}_{\mu})-%%@
% 2A_{\mu}Z^{0}_{\mu}W^{+}_{\nu}W^{-}_{\nu}]
% -g\alpha[H^3+H\phi^{0}\phi^{0}+2H\phi^{+}\phi^{-}]
% -\frac{1}{8}g^{2}\alpha_{h}[H^4+(\phi^{0})^{4}+4(\phi^{+}\phi^{-})^{2}+4(\phi^{0})^{2}
% \phi^{+}\phi^{-}+4H^{2}\phi^{+}\phi^{-}+2(\phi^{0})^{2}H^{2}]
% -gMW^{+}_{\mu}W^{-}_{\mu}H-\frac{1}{2}g\frac{M}{c^{2}_{w}}Z^{0}_{\mu}Z^{0}_{\mu}H
% -\frac{1}{2}ig[W^{+}_{\mu}(\phi^{0}\partial_{\mu}\phi^{-}-\phi^{-}\partial_{\mu}\phi^{0})
% -W^{-}_{\mu}(\phi^{0}\partial_{\mu}\phi^{+}-\phi^{+}\partial_{\mu}\phi^{0})]
% +\frac{1}{2}g[W^{+}_{\mu}(H\partial_{\mu}\phi^{-}-\phi^{-}\partial_{\mu}H)
% -W^{-}_{\mu}(H\partial_{\mu}\phi^{+}-\phi^{+}\partial_{\mu}H)]
% +\frac{1}{2}g\frac{1}{c_{w}}(Z^{0}_{\mu}(H\partial_{\mu}\phi^{0}-\phi^{0}\partial_{\mu}H)
% -ig\frac{s^{2}_{w}}{c_{w}}MZ^{0}_{\mu}(W^{+}_{\mu}\phi^{-}-W^{-}_{\mu}\phi^{+})
% +igs_{w}MA_{\mu}(W^{+}_{\mu}\phi^{-}-W^{-}_{\mu}\phi^{+})
% -ig\frac{1-2c^{2}_{w}}{2c_{w}}Z^{0}_{\mu}(\phi^{+}\partial_{\mu}\phi^{-}-\phi^{-%%@
% }\partial_{\mu}\phi^{+})
% +igs_{w}A_{\mu}(\phi^{+}\partial_{\mu}\phi^{-}-\phi^{-}\partial_{\mu}\phi^{+})
% -\frac{1}{4}g^{2}W^{+}_{\mu}W^{-}_{\mu}[H^{2}+(\phi^{0})^{2}+2\phi^{+}\phi^{-}]
% -\frac{1}{4}g^{2}\frac{1}{c^{2}_{w}}Z^{0}_{\mu}Z^{0}_{\mu}[H^{2}+(\phi^{0})^{2}+2(2s^{2}_{w}-%%@
% 1)^{2}\phi^{+}\phi^{-}]
% -\frac{1}{2}g^{2}\frac{s^{2}_{w}}{c_{w}}Z^{0}_{\mu}\phi^{0}(W^{+}_{\mu}\phi^{-}+W^{-%%@
% }_{\mu}\phi^{+})
% -\frac{1}{2}ig^{2}\frac{s^{2}_{w}}{c_{w}}Z^{0}_{\mu}H(W^{+}_{\mu}\phi^{-}-W^{-}_{\mu}\phi^{+})
% +\frac{1}{2}g^{2}s_{w}A_{\mu}\phi^{0}(W^{+}_{\mu}\phi^{-}+W^{-}_{\mu}\phi^{+})
% +\frac{1}{2}ig^{2}s_{w}A_{\mu}H(W^{+}_{\mu}\phi^{-}-W^{-}_{\mu}\phi^{+})
% -g^{2}\frac{s_{w}}{c_{w}}(2c^{2}_{w}-1)Z^{0}_{\mu}A_{\mu}\phi^{+}\phi^{-}-%%@
% g^{1}s^{2}_{w}A_{\mu}A_{\mu}\phi^{+}\phi^{-}
% -\bar{e}^{\lambda}(\gamma\partial+m^{\lambda}_{e})e^{\lambda}
% -\bar{\nu}^{\lambda}\gamma\partial\nu^{\lambda}
% -\bar{u}^{\lambda}_{j}(\gamma\partial+m^{\lambda}_{u})u^{\lambda}_{j}
% -\bar{d}^{\lambda}_{j}(\gamma\partial+m^{\lambda}_{d})d^{\lambda}_{j}
% +igs_{w}A_{\mu}[-(\bar{e}^{\lambda}\gamma^{\mu}
% e^{\lambda})+\frac{2}{3}(\bar{u}^{\lambda}_{j}\gamma^{\mu} %%@
% u^{\lambda}_{j})-\frac{1}{3}(\bar{d}^{\lambda}_{j}\gamma^{\mu} 
% d^{\lambda}_{j})]
% +\frac{ig}{4c_{w}}Z^{0}_{\mu}
% [(\bar{\nu}^{\lambda}\gamma^{\mu}(1+\gamma^{5})\nu^{\lambda})+
% (\bar{e}^{\lambda}\gamma^{\mu}(4s^{2}_{w}-1-\gamma^{5})e^{\lambda})+
% (\bar{u}^{\lambda}_{j}\gamma^{\mu}(\frac{4}{3}s^{2}_{w}-1-\gamma^{5})u^{\lambda}_{j})+
% (\bar{d}^{\lambda}_{j}\gamma^{\mu}(1-\frac{8}{3}s^{2}_{w}-\gamma^{5})d^{\lambda}_{j})]
% +\frac{ig}{2\sqrt{2}}W^{+}_{\mu}[(\bar{\nu}^{\lambda}\gamma^{\mu}(1+\gamma^{5})e^{\lambda})
% +(\bar{u}^{\lambda}_{j}\gamma^{\mu}(1+\gamma^{5})C_{\lambda\kappa}d^{\kappa}_{j})]
% +\frac{ig}{2\sqrt{2}}W^{-}_{\mu}[(\bar{e}^{\lambda}\gamma^{\mu}(1+\gamma^{5})\nu^{\lambda})
% +(\bar{d}^{\kappa}_{j}C^{\dagger}_{\lambda\kappa}\gamma^{\mu}(1+\gamma^{5})u^{\lambda}_{j})]
% +\frac{ig}{2\sqrt{2}}\frac{m^{\lambda}_{e}}{M}
% [-\phi^{+}(\bar{\nu}^{\lambda}(1-\gamma^{5})e^{\lambda})
% +\phi^{-}(\bar{e}^{\lambda}(1+\gamma^{5})\nu^{\lambda})]
% -\frac{g}{2}\frac{m^{\lambda}_{e}}{M}[H(\bar{e}^{\lambda}e^{\lambda})
% +i\phi^{0}(\bar{e}^{\lambda}\gamma^{5}e^{\lambda})]
% +\frac{ig}{2M\sqrt{2}}\phi^{+}
% [-m^{\kappa}_{d}(\bar{u}^{\lambda}_{j}C_{\lambda\kappa}(1-\gamma^{5})d^{\kappa}_{j})
% +m^{\lambda}_{u}(\bar{u}^{\lambda}_{j}C_{\lambda\kappa}(1+\gamma^{5})d^{\kappa}_{j}]
% +\frac{ig}{2M\sqrt{2}}\phi^{-}
% [m^{\lambda}_{d}(\bar{d}^{\lambda}_{j}C^{\dagger}_{\lambda\kappa}(1+\gamma^{5})u^{\kappa}_{j})
% -m^{\kappa}_{u}(\bar{d}^{\lambda}_{j}C^{\dagger}_{\lambda\kappa}(1-\gamma^{5})u^{\kappa}_{j}]
% -\frac{g}{2}\frac{m^{\lambda}_{u}}{M}H(\bar{u}^{\lambda}_{j}u^{\lambda}_{j})
% -\frac{g}{2}\frac{m^{\lambda}_{d}}{M}H(\bar{d}^{\lambda}_{j}d^{\lambda}_{j})
% +\frac{ig}{2}\frac{m^{\lambda}_{u}}{M}\phi^{0}(\bar{u}^{\lambda}_{j}\gamma^{5}u^{\lambda}_{j})
% -\frac{ig}{2}\frac{m^{\lambda}_{d}}{M}\phi^{0}(\bar{d}^{\lambda}_{j}\gamma^{5}d^{\lambda}_{j})
% +\bar{X}^{+}(\partial^{2}-M^{2})X^{+}+\bar{X}^{-}(\partial^{2}-M^{2})X^{-}
% +\bar{X}^{0}(\partial^{2}-\frac{M^{2}}{c^{2}_{w}})X^{0}+\bar{Y}\partial^{2}Y
% +igc_{w}W^{+}_{\mu}(\partial_{\mu}\bar{X}^{0}X^{-}-\partial_{\mu}\bar{X}^{+}X^{0})
% +igs_{w}W^{+}_{\mu}(\partial_{\mu}\bar{Y}X^{-}-\partial_{\mu}\bar{X}^{+}Y)
% +igc_{w}W^{-}_{\mu}(\partial_{\mu}\bar{X}^{-}X^{0}-\partial_{\mu}\bar{X}^{0}X^{+})
% +igs_{w}W^{-}_{\mu}(\partial_{\mu}\bar{X}^{-}Y-\partial_{\mu}\bar{Y}X^{+})
% +igc_{w}Z^{0}_{\mu}(\partial_{\mu}\bar{X}^{+}X^{+}-\partial_{\mu}\bar{X}^{-}X^{-})
% +igs_{w}A_{\mu}(\partial_{\mu}\bar{X}^{+}X^{+}-\partial_{\mu}\bar{X}^{-}X^{-})
% -\frac{1}{2}gM[\bar{X}^{+}X^{+}H+\bar{X}^{-}X^{-}H+\frac{1}{c^{2}_{w}}\bar{X}^{0}X^{0}H]
% +\frac{1-2c^{2}_{w}}{2c_{w}}igM[\bar{X}^{+}X^{0}\phi^{+}-\bar{X}^{-}X^{0}\phi^{-}]
% +\frac{1}{2c_{w}}igM[\bar{X}^{0}X^{-}\phi^{+}-\bar{X}^{0}X^{+}\phi^{-}]
% +igMs_{w}[\bar{X}^{0}X^{-}\phi^{+}-\bar{X}^{0}X^{+}\phi^{-}]
% +\frac{1}{2}igM[\bar{X}^{+}X^{+}\phi^{0}-\bar{X}^{-}X^{-}\phi^{0}]
% \end{math}
% \end{center}



%%% Local Variables: 
%%% mode: latex
%%% TeX-master: "../thesis"
%%% End: 
