%!TEX root = ../thesis.tex

\chapter{Theoretical Background}
\label{c:theory}
%or phenomenology?

\ifpdf
    \graphicspath{{02_Theory/plots/}}
\else
    \graphicspath{{02_Theory/plots/EPS/}{02_Theory/plots/}}
\fi

\section{The Standard Model of Particle Physics}
\label{s:SM}
The Standard Model (SM) is a quantum field theory which is currently the most accurate description of matter and all
known fundamental interactions, with the exception of gravity. It was developed in the 20th century as a combination of
two complementary field theories, the Glashow-Weinberg-Salam (GWS) \autocite{Glashow, Weinberg, Salam} theory of
electroweak interaction, and Quantum Chromodynamics (QCD) theory of strong interaction.

The fundamental particles in the Standard Model are the three generations of fermions, shown in
Table~\ref{tab:SM_fermions}, interacting via gauge bosons of three fundamental interactions, presented in
Table~\ref{tab:SM_forces}. Fermions, subdivided into quarks and leptons, are the building blocks of matter in the
observable universe. Their interaction via exchanging the gauge bosons allows formation of hadrons and atoms. The
difference between quarks and leptons mainly lies in the way they interact: quarks can interact via strong,
electromagnetic and weak forces, whereas leptons are only subject to electromagnetic and weak interactions. Each of
these particles also has a corresponding antiparticle with the same mass, but opposite charge.

There are six flavours of leptons in the SM, making up three generations: electron ($e$), muon ($\mu$) and tau ($\tau$)
leptons with their corresponding neutrinos. Electrons, muons and taus are massive and charged particles, interacting via
electromagnetic and weak forces, described in Section~\ref{ss:electroweak_theory}. Neutrinos, on the other hand, do not
carry an electric charge, and therefore are only subject to the weak force, which makes them extremely difficult to
detect as they barely interact with matter. In the classic version of Standard Model neutrinos are considered to be
massless, however, the observation of neutrino oscillations between different flavours \autocite{neutrino_oscillations}
implies that neutrinos are in fact massive. This is one of the deficiencies of the Standard Model, giving rise to
possible extensions elaborated in Section~\ref{s:BSM}.

The quarks also exist in three generations, and are represented by up and down (\cPqu, \cPqd), charm and strange (\cPqc,
\cPqs), and bottom and top (\cPqb, \cPqt) flavours. Together with the mediators of the strong interaction (gluons),
these particles possess another physical property called colour charge. Due to a phenomenon of colour confinement, they
can only exist as constituents of composite particles -- hadrons. More detail on the QCD theory of strong interaction is
presented in Section~\ref{ss:QCD_theory}. The top quark, however, is too heavy hadronise as it has an extremely short
lifetime of \SI{\sim e-25}{\s}, which is an order of magnitude shorter than the timescale of strong interactions. This
and other properties of the top quark will be discussed in Section~\ref{s:top_quak_physics}.

The last essential constituent of the Standard Model is the recently discovered Higgs boson
\autocite{ATLAS_higgs_observation, CMS_higgs_observation}. The Higgs field is responsible for electroweak symmetry
breaking of $SU(2)_L \times U(1)_Y$ group and therefore acquisition of mass by vector bosons. This mechanism is
described in more detail in Section~\ref{ss:electroweak_symmetry_breaking}.

\begin{table}[htbp]
\centering
\caption[Fundamental fermions of the Standard Model]{Three generations of fundamental fermions of the Standard Model
with charge and mass properties quoted from the 2012 edition of Review of Particle Physics with 2013 partial update for
the 2014 edition \autocite{PDG, world_top_mass_combination}.}
\label{tab:SM_fermions}
\resizebox{\textwidth}{!} {
\begin{tabular}{@{}_c|^c|^c|^c|^c|^c|^c@{}}
 \toprule
 \multirow{2}{*}[-2pt]{Generation} & \multicolumn{3}{c|}{Leptons} & \multicolumn{3}{c}{Quarks} \\
 \cmidrule{2-7}
  & Flavour & Charge & Mass [\si{\MeV}] & Flavour & Charge & Mass [\si{\MeV}] \\
 \midrule
 \multirow{2}{*}{1} & $e$ & \num{-1} & \num{0.511} & \cPqu & \num{+2/3} & $2.3^{+0.7}_{-0.5}$ \\
                    & $\nu_e$ & \num{0} & $<\num{2e-6}$ & \cPqd & \num{-1/3} & $4.8^{+0.5}_{-0.3}$ \\
 \midrule
 \multirow{2}{*}{2} & $\mu$ & \num{-1} & \num{105.66} & \cPqc & \num{+2/3} & $(1.29^{+0.05}_{-0.11}) \times 10^3$ \\
                    & $\nu_\mu$ & \num{0} & $<\num{0.19}$ & \cPqs & \num{-1/3} & $95 \pm 5$ \\
 \midrule
 \multirow{2}{*}{3} & $\tau$ & \num{-1} & $1776.82 \pm 0.16$ & \cPqt & \num{+2/3} & $(173.34 \pm 0.76) \times 10^3$ \\
                    & $\nu_\tau$ & \num{0} & $<\num{18.2}$ & \cPqb & \num{-1/3} & $(4.18 \pm 0.03) \times 10^3$ \\

%http://pdg.lbl.gov/2013/tables/rpp2013-sum-quarks.pdf

\bottomrule
\end{tabular}}
\end{table}

\begin{table}[thbp]
\centering
\caption[Fundamental forces and corresponding gauge bosons with their properties]{Fundamental forces and corresponding
gauge bosons with their properties \autocite{PDG}. The gravitation is the only fundamental interaction not desribed by
the Standard Model.}
\label{tab:SM_forces} 
\begin{tabular}{@{}_l|^c|^c|^c|^c|^c@{}}
 \toprule
 Force & Gauge Boson & Charge & Spin & Mass \si{[\GeV]} & Range\\ 
 \midrule
 electromagnetic  & photon ($\gamma$) & 0 & 1 & 0 & $\infty$\\
 \midrule
 \multirow{2}{*}{weak} & $\W^\pm$ & \num{\pm1} & \multirow{2}{*}{\num{1}} & $80.385 \pm 0.015$ & \multirow{2}{*}{\SI{d-18}{\metre}}\\   
                       & $\Z^0$   & \num{0} &                             & $91.1876 \pm 0.0021$ &                                 \\
 \midrule
 strong  & 8 gluons (\cPg) & 0 & 1 & 0 & \SI{d-15}{\metre} \\
 \midrule
 gravitational  & graviton (\cPG) & 0 & 2 & 0 & $\infty$\\
\bottomrule
\end{tabular}
\end{table}

\newpage
\subsection{Gauge Principle}
\label{ss:gauge_principle}
As it was mentioned, the Standard Model is a unification of two gauge theories describing electroweak and strong
interactions. The GWS model, a unified theory of electromagnetic and weak interactions, is based on a gauge group of
$SU(2)_L \times U(1)_Y$, whereas the QCD theory has a gauge symmetry of $SU(3)_C$. Therefore, the Standard Model is also
a gauge theory with the symmetry group of $SU(3)_C \times SU(2)_L \times U(1)_Y$. A gauge theory is a theory that is
invariant under a set of local transformations, i.e.\ transformations with space-time dependent parameters. To explain
this concept, let us consider a Lagrangian density of a free Dirac field $\psi$, describing free-moving fermions:
\begin{equation}
\calL = \bar{\psi}(i \gamma^\mu	\partial_\mu - m) \psi
\label{eq:dirac_field_L}
\end{equation}

Clearly, it is invariant under the following phase transformation:
\begin{equation}
\psi \rightarrow \psi' = e^{i \theta} \psi, ~\bar{\psi} \rightarrow \bar{\psi} = e^{-i \theta} \bar{\psi},
\end{equation}
since in the combination $\bar{\psi}\psi$ the exponential factors cancel out ($e^{i \theta} e^{-i \theta} = 1$). This
transformation is called a \textit{global} phase transformation, since the phase $i \theta$ is space-time independent.
The \textit{local} phase transformation, on the other hand, corresponds to the phase $i \theta(x_\mu)$ which is
different at each space-time point:
\begin{equation}
\psi \rightarrow \psi' = e^{i \theta(x)} \psi, ~\bar{\psi} \rightarrow \bar{\psi} = e^{-i \theta(x)} \bar{\psi}.
\end{equation}
Here the space-time indices are dropped on $x_\mu \equiv x$ for clarity. One can see that the Lagrangian is no longer
invariant under such transformation, since
\begin{equation}
\partial_\mu (e^{i \theta(x)} \psi) = i (\partial_\mu \theta(x))  e^{i \theta(x)} \psi + e^{i \theta(x)} \partial_\mu \psi,
\end{equation}
and therefore
\begin{equation}
\calL \rightarrow \calL - [\partial_\mu \theta(x)] \bar{\psi} \gamma^\mu \psi.
\end{equation}

The space-time dependent local transformations are called gauge transformations. In order to restore gauge invariance
under such transformations, we need to substitute the ordinary derivative $\partial_\mu$ with the \textit{covariant}
derivative $D_\mu$:
\begin{equation}
\partial_\mu \rightarrow D_\mu = \partial_\mu + i q A_\mu,
\end{equation}
where $A_\mu$ is a new vector field which transforms under a local gauge transformation as:
\begin{equation}
A_\mu \rightarrow A'_\mu = A_\mu - \frac{1}{q} \partial_\mu \theta(x).
\end{equation}

Together with the kinetic energy term of the field-strength tensor defined as
\begin{equation}
F_{\mu\nu} \equiv \partial_\mu A_\nu - \partial_\nu A_\mu,
\end{equation}
we obtain the full gauge invariant QED Lagrangian:
\begin{equation}
\calL_\textrm{QED} = \bar{\psi}(i \gamma^\mu (\partial^\mu + i q A_\mu) - m) \psi - \frac{1}{4}F_{\mu\nu} F^{\mu\nu}.
%\bar{\psi}(i \gamma^\mu \partial^\mu -m) \psi - q \bar{\psi}\gamma^\mu A_\mu \psi - \frac{1}{4}F_{\mu\nu} F^{\mu\nu}.
\end{equation}

The requirement of local gauge invariance necessitates the gauge field $A_\mu$ to be massless, since the term
proportional to $A_\mu A^\mu$ is not invariant under local gauge transformations and therefore has to be excluded from
the Lagrangian. In case of QED, the gauge field $A_\mu$ represents the photon field, and the Lagrangian describes the
interaction of Dirac fields (electrons and positrons) with Maxwell fields (photons).

The idea of requiring a local gauge invariance by introducing additional fields in the Lagrangian to make it covariant
with respect to an extended group of local transformations is an important procedure in particle physics, referred to as
gauge principle. In the example above, the local phase transformation can be thought of as multiplication of the Dirac
field $\psi$ by a $1 \times 1$ unitary matrix ($\psi \rightarrow U \psi$, where $U^{\dag}U = 1$). Here $U = e^{i
\theta(x)}$, and all such transformations form an Abelian Lie group of $U(1)$, since any two its elements commute:
\begin{equation}
[e^{i \theta(x)},~e^{i\theta(x')}] = 0.
\end{equation}

The same strategy of requiring the global invariance to hold locally can be generalised to any $SU(N)$ group: in 1950s
Yang and Mills produced the theory of $SU(2)$ gauge fields, and later on the idea was extended to $SU(3)$ group giving
rise to Quantum Chromodynamics. As a matter of fact, all of the fundamental interactions in the Standard Model are
introduced using the same gauge principle. However, in order to preserve gauge invariance, the introduced gauge fields
are required to be massless, which clearly contradicts the existence of massive vector bosons. This problem is solved by
the Brout-Englert-Higgs mechanism of spontaneous gauge symmetry breaking
(Section~\ref{ss:electroweak_symmetry_breaking}).


\subsection{Electroweak theory}
\label{ss:electroweak_theory}




\subsection{Quantum Chromodynamics}
\label{ss:QCD_theory}

\subsection{Electroweak symmetry breaking}
\label{ss:electroweak_symmetry_breaking}
%or englert-higgs mechanism?

\section{Top Quark Physics within the Standard Model}
\label{s:top_quak_physics}

\subsection{Top quark production at the LHC}
\label{ss:top_production}

\subsection{Top quark decay}
\label{ss:top_decay}

\subsection{Background processes}
\label{ss:backgrounds}

\subsection{Top quark mass}
\label{ss:top_mass}

\subsection{Importance of top quark physics}
\label{ss:importance}

\section{Physics Beyond The Standard Model}
\label{s:BSM}
% There are simple extensions to the Standard Model describing massive neutrinos and keeping the local symmetry of weak
% interactions.

\subsection{Supersymmetry}
\label{ss:SUSY}

\subsection{Extra dimensions}
\label{ss:extra_dimensions}

% \subsection{Standard Model Lagrangian}
% Because why the hell not:

% \begin{center}
% \begin{math}
% \calL = -\frac{1}{2}\partial_{\nu}g^{a}_{\mu}\partial_{\nu}g^{a}_{\mu}
% -g_{s}f^{abc}\partial_{\mu}g^{a}_{\nu}g^{b}_{\mu}g^{c}_{\nu}
% -\frac{1}{4}g^{2}_{s}f^{abc}f^{ade}g^{b}_{\mu}g^{c}_{\nu}g^{d}_{\mu}g^{e}_{\nu}
% +\frac{1}{2}ig^{2}_{s}(\bar{q}^{\sigma}_{i}\gamma^{\mu}q^{\sigma}_{j})g^{a}_{\mu}
% +\bar{G}^{a}\partial^{2}G^{a}+g_{s}f^{abc}\partial_{\mu}\bar{G}^{a}G^{b}g^{c}_{\mu}
% -\partial_{\nu}W^{+}_{\mu}\partial_{\nu}W^{-}_{\mu}-M^{2}W^{+}_{\mu}W^{-}_{\mu}
% -\frac{1}{2}\partial_{\nu}Z^{0}_{\mu}\partial_{\nu}Z^{0}_{\mu}-\frac{1}{2c^{2}_{w}}
% M^{2}Z^{0}_{\mu}Z^{0}_{\mu}
% -\frac{1}{2}\partial_{\mu}A_{\nu}\partial_{\mu}A_{\nu}
% -\frac{1}{2}\partial_{\mu}H\partial_{\mu}H-\frac{1}{2}m^{2}_{h}H^{2}
% -\partial_{\mu}\phi^{+}\partial_{\mu}\phi^{-}-M^{2}\phi^{+}\phi^{-}
% -\frac{1}{2}\partial_{\mu}\phi^{0}\partial_{\mu}\phi^{0}-\frac{1}{2c^{2}_{w}}M\phi^{0}\phi^{0}
% -\beta_{h}[\frac{2M^{2}}{g^{2}}+\frac{2M}{g}H+\frac{1}{2}(H^{2}+\phi^{0}\phi^{0}+2\phi^{+}\phi^{-%%@
% })]+\frac{2M^{4}}{g^{2}}\alpha_{h}
% -igc_{w}[\partial_{\nu}Z^{0}_{\mu}(W^{+}_{\mu}W^{-}_{\nu}-W^{+}_{\nu}W^{-}_{\mu})
% -Z^{0}_{\nu}(W^{+}_{\mu}\partial_{\nu}W^{-}_{\mu}-W^{-}_{\mu}\partial_{\nu}W^{+}_{\mu})
% +Z^{0}_{\mu}(W^{+}_{\nu}\partial_{\nu}W^{-}_{\mu}-W^{-}_{\nu}\partial_{\nu}W^{+}_{\mu})]
% -igs_{w}[\partial_{\nu}A_{\mu}(W^{+}_{\mu}W^{-}_{\nu}-W^{+}_{\nu}W^{-}_{\mu})
% -A_{\nu}(W^{+}_{\mu}\partial_{\nu}W^{-}_{\mu}-W^{-}_{\mu}\partial_{\nu}W^{+}_{\mu})
% +A_{\mu}(W^{+}_{\nu}\partial_{\nu}W^{-}_{\mu}-W^{-}_{\nu}\partial_{\nu}W^{+}_{\mu})]
% -\frac{1}{2}g^{2}W^{+}_{\mu}W^{-}_{\mu}W^{+}_{\nu}W^{-}_{\nu}+\frac{1}{2}g^{2}
% W^{+}_{\mu}W^{-}_{\nu}W^{+}_{\mu}W^{-}_{\nu}
% +g^2c^{2}_{w}(Z^{0}_{\mu}W^{+}_{\mu}Z^{0}_{\nu}W^{-}_{\nu}-Z^{0}_{\mu}Z^{0}_{\mu}W^{+}_{\nu}
% W^{-}_{\nu})
% +g^2s^{2}_{w}(A_{\mu}W^{+}_{\mu}A_{\nu}W^{-}_{\nu}-A_{\mu}A_{\mu}W^{+}_{\nu}
% W^{-}_{\nu})
% +g^{2}s_{w}c_{w}[A_{\mu}Z^{0}_{\nu}(W^{+}_{\mu}W^{-}_{\nu}-W^{+}_{\nu}W^{-}_{\mu})-%%@
% 2A_{\mu}Z^{0}_{\mu}W^{+}_{\nu}W^{-}_{\nu}]
% -g\alpha[H^3+H\phi^{0}\phi^{0}+2H\phi^{+}\phi^{-}]
% -\frac{1}{8}g^{2}\alpha_{h}[H^4+(\phi^{0})^{4}+4(\phi^{+}\phi^{-})^{2}+4(\phi^{0})^{2}
% \phi^{+}\phi^{-}+4H^{2}\phi^{+}\phi^{-}+2(\phi^{0})^{2}H^{2}]
% -gMW^{+}_{\mu}W^{-}_{\mu}H-\frac{1}{2}g\frac{M}{c^{2}_{w}}Z^{0}_{\mu}Z^{0}_{\mu}H
% -\frac{1}{2}ig[W^{+}_{\mu}(\phi^{0}\partial_{\mu}\phi^{-}-\phi^{-}\partial_{\mu}\phi^{0})
% -W^{-}_{\mu}(\phi^{0}\partial_{\mu}\phi^{+}-\phi^{+}\partial_{\mu}\phi^{0})]
% +\frac{1}{2}g[W^{+}_{\mu}(H\partial_{\mu}\phi^{-}-\phi^{-}\partial_{\mu}H)
% -W^{-}_{\mu}(H\partial_{\mu}\phi^{+}-\phi^{+}\partial_{\mu}H)]
% +\frac{1}{2}g\frac{1}{c_{w}}(Z^{0}_{\mu}(H\partial_{\mu}\phi^{0}-\phi^{0}\partial_{\mu}H)
% -ig\frac{s^{2}_{w}}{c_{w}}MZ^{0}_{\mu}(W^{+}_{\mu}\phi^{-}-W^{-}_{\mu}\phi^{+})
% +igs_{w}MA_{\mu}(W^{+}_{\mu}\phi^{-}-W^{-}_{\mu}\phi^{+})
% -ig\frac{1-2c^{2}_{w}}{2c_{w}}Z^{0}_{\mu}(\phi^{+}\partial_{\mu}\phi^{-}-\phi^{-%%@
% }\partial_{\mu}\phi^{+})
% +igs_{w}A_{\mu}(\phi^{+}\partial_{\mu}\phi^{-}-\phi^{-}\partial_{\mu}\phi^{+})
% -\frac{1}{4}g^{2}W^{+}_{\mu}W^{-}_{\mu}[H^{2}+(\phi^{0})^{2}+2\phi^{+}\phi^{-}]
% -\frac{1}{4}g^{2}\frac{1}{c^{2}_{w}}Z^{0}_{\mu}Z^{0}_{\mu}[H^{2}+(\phi^{0})^{2}+2(2s^{2}_{w}-%%@
% 1)^{2}\phi^{+}\phi^{-}]
% -\frac{1}{2}g^{2}\frac{s^{2}_{w}}{c_{w}}Z^{0}_{\mu}\phi^{0}(W^{+}_{\mu}\phi^{-}+W^{-%%@
% }_{\mu}\phi^{+})
% -\frac{1}{2}ig^{2}\frac{s^{2}_{w}}{c_{w}}Z^{0}_{\mu}H(W^{+}_{\mu}\phi^{-}-W^{-}_{\mu}\phi^{+})
% +\frac{1}{2}g^{2}s_{w}A_{\mu}\phi^{0}(W^{+}_{\mu}\phi^{-}+W^{-}_{\mu}\phi^{+})
% +\frac{1}{2}ig^{2}s_{w}A_{\mu}H(W^{+}_{\mu}\phi^{-}-W^{-}_{\mu}\phi^{+})
% -g^{2}\frac{s_{w}}{c_{w}}(2c^{2}_{w}-1)Z^{0}_{\mu}A_{\mu}\phi^{+}\phi^{-}-%%@
% g^{1}s^{2}_{w}A_{\mu}A_{\mu}\phi^{+}\phi^{-}
% -\bar{e}^{\lambda}(\gamma\partial+m^{\lambda}_{e})e^{\lambda}
% -\bar{\nu}^{\lambda}\gamma\partial\nu^{\lambda}
% -\bar{u}^{\lambda}_{j}(\gamma\partial+m^{\lambda}_{u})u^{\lambda}_{j}
% -\bar{d}^{\lambda}_{j}(\gamma\partial+m^{\lambda}_{d})d^{\lambda}_{j}
% +igs_{w}A_{\mu}[-(\bar{e}^{\lambda}\gamma^{\mu}
% e^{\lambda})+\frac{2}{3}(\bar{u}^{\lambda}_{j}\gamma^{\mu} %%@
% u^{\lambda}_{j})-\frac{1}{3}(\bar{d}^{\lambda}_{j}\gamma^{\mu} 
% d^{\lambda}_{j})]
% +\frac{ig}{4c_{w}}Z^{0}_{\mu}
% [(\bar{\nu}^{\lambda}\gamma^{\mu}(1+\gamma^{5})\nu^{\lambda})+
% (\bar{e}^{\lambda}\gamma^{\mu}(4s^{2}_{w}-1-\gamma^{5})e^{\lambda})+
% (\bar{u}^{\lambda}_{j}\gamma^{\mu}(\frac{4}{3}s^{2}_{w}-1-\gamma^{5})u^{\lambda}_{j})+
% (\bar{d}^{\lambda}_{j}\gamma^{\mu}(1-\frac{8}{3}s^{2}_{w}-\gamma^{5})d^{\lambda}_{j})]
% +\frac{ig}{2\sqrt{2}}W^{+}_{\mu}[(\bar{\nu}^{\lambda}\gamma^{\mu}(1+\gamma^{5})e^{\lambda})
% +(\bar{u}^{\lambda}_{j}\gamma^{\mu}(1+\gamma^{5})C_{\lambda\kappa}d^{\kappa}_{j})]
% +\frac{ig}{2\sqrt{2}}W^{-}_{\mu}[(\bar{e}^{\lambda}\gamma^{\mu}(1+\gamma^{5})\nu^{\lambda})
% +(\bar{d}^{\kappa}_{j}C^{\dagger}_{\lambda\kappa}\gamma^{\mu}(1+\gamma^{5})u^{\lambda}_{j})]
% +\frac{ig}{2\sqrt{2}}\frac{m^{\lambda}_{e}}{M}
% [-\phi^{+}(\bar{\nu}^{\lambda}(1-\gamma^{5})e^{\lambda})
% +\phi^{-}(\bar{e}^{\lambda}(1+\gamma^{5})\nu^{\lambda})]
% -\frac{g}{2}\frac{m^{\lambda}_{e}}{M}[H(\bar{e}^{\lambda}e^{\lambda})
% +i\phi^{0}(\bar{e}^{\lambda}\gamma^{5}e^{\lambda})]
% +\frac{ig}{2M\sqrt{2}}\phi^{+}
% [-m^{\kappa}_{d}(\bar{u}^{\lambda}_{j}C_{\lambda\kappa}(1-\gamma^{5})d^{\kappa}_{j})
% +m^{\lambda}_{u}(\bar{u}^{\lambda}_{j}C_{\lambda\kappa}(1+\gamma^{5})d^{\kappa}_{j}]
% +\frac{ig}{2M\sqrt{2}}\phi^{-}
% [m^{\lambda}_{d}(\bar{d}^{\lambda}_{j}C^{\dagger}_{\lambda\kappa}(1+\gamma^{5})u^{\kappa}_{j})
% -m^{\kappa}_{u}(\bar{d}^{\lambda}_{j}C^{\dagger}_{\lambda\kappa}(1-\gamma^{5})u^{\kappa}_{j}]
% -\frac{g}{2}\frac{m^{\lambda}_{u}}{M}H(\bar{u}^{\lambda}_{j}u^{\lambda}_{j})
% -\frac{g}{2}\frac{m^{\lambda}_{d}}{M}H(\bar{d}^{\lambda}_{j}d^{\lambda}_{j})
% +\frac{ig}{2}\frac{m^{\lambda}_{u}}{M}\phi^{0}(\bar{u}^{\lambda}_{j}\gamma^{5}u^{\lambda}_{j})
% -\frac{ig}{2}\frac{m^{\lambda}_{d}}{M}\phi^{0}(\bar{d}^{\lambda}_{j}\gamma^{5}d^{\lambda}_{j})
% +\bar{X}^{+}(\partial^{2}-M^{2})X^{+}+\bar{X}^{-}(\partial^{2}-M^{2})X^{-}
% +\bar{X}^{0}(\partial^{2}-\frac{M^{2}}{c^{2}_{w}})X^{0}+\bar{Y}\partial^{2}Y
% +igc_{w}W^{+}_{\mu}(\partial_{\mu}\bar{X}^{0}X^{-}-\partial_{\mu}\bar{X}^{+}X^{0})
% +igs_{w}W^{+}_{\mu}(\partial_{\mu}\bar{Y}X^{-}-\partial_{\mu}\bar{X}^{+}Y)
% +igc_{w}W^{-}_{\mu}(\partial_{\mu}\bar{X}^{-}X^{0}-\partial_{\mu}\bar{X}^{0}X^{+})
% +igs_{w}W^{-}_{\mu}(\partial_{\mu}\bar{X}^{-}Y-\partial_{\mu}\bar{Y}X^{+})
% +igc_{w}Z^{0}_{\mu}(\partial_{\mu}\bar{X}^{+}X^{+}-\partial_{\mu}\bar{X}^{-}X^{-})
% +igs_{w}A_{\mu}(\partial_{\mu}\bar{X}^{+}X^{+}-\partial_{\mu}\bar{X}^{-}X^{-})
% -\frac{1}{2}gM[\bar{X}^{+}X^{+}H+\bar{X}^{-}X^{-}H+\frac{1}{c^{2}_{w}}\bar{X}^{0}X^{0}H]
% +\frac{1-2c^{2}_{w}}{2c_{w}}igM[\bar{X}^{+}X^{0}\phi^{+}-\bar{X}^{-}X^{0}\phi^{-}]
% +\frac{1}{2c_{w}}igM[\bar{X}^{0}X^{-}\phi^{+}-\bar{X}^{0}X^{+}\phi^{-}]
% +igMs_{w}[\bar{X}^{0}X^{-}\phi^{+}-\bar{X}^{0}X^{+}\phi^{-}]
% +\frac{1}{2}igM[\bar{X}^{+}X^{+}\phi^{0}-\bar{X}^{-}X^{-}\phi^{0}]
% \end{math}
% \end{center}



%%% Local Variables: 
%%% mode: latex
%%% TeX-master: "../thesis"
%%% End: 
