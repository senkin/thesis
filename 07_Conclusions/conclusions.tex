%!TEX root = ../thesis.tex

\def\baselinestretch{1}
\chapter{Conclusions}
\label{c:conclusions}
\ifpdf
    \graphicspath{{Conclusions/plots/}}
\else
    \graphicspath{{Conclusions/plots/EPS/}{Conclusions/plots/}}
\fi

\def\baselinestretch{1.66}

This thesis has covered the author's and collaborators' research carried out in the top quark sector of particle
physics. Two major analysis contributions, namely the top quark mass measurement in the electron plus jets channel, and
the \ttbar differential cross section measurement in the electron and muon plus jets channels, were presented in detail.
Additionally, the service work performed by the author for the CMS collaboration -- the high-level trigger development
for top quark physics -- was also covered. The triggers developed were used not only in the mass analysis presented, but
also in other CMS top quark physics analyses using the electron plus jets \ttbar events collected at a centre of mass
energy of $\sqrt{s} = \SI{7}{\TeV}$.

The top quark mass measurement was performed using the full 2011 dataset recorded by the CMS detector with a total
integrated luminosity of approximately \SI{5.0}{\fbinv}. The analysis served as a complementary cross-check of the main
CMS top quark mass measurement in the lepton plus jets channel, which remains the most precise single measurement of the
top quark mass to date. The mass extraction technique and the kinematic fit procedure, common between two analyses, were
described in detail. The analyses results are consistent with each other, as well as with the latest combination result
from the Tevatron and the LHC experiments.

The differential cross section measurement of top quark pair production with respect to event-level distributions is
based on 2012 data collected by CMS at a centre of mass energy of $\sqrt{s} = \SI{8}{\TeV}$, corresponding to an
integrated luminosity of \SI{19.7}{\fbinv}. The analysis selected semileptonic \ttbar events with a single isolated
highly-energetic electron or muon, and at least four jets in the final state. The number of \ttbar events was determined
using a template fit method, and corrected for misidentification, detector resolution and acceptance effects using the
SVD unfolding technique. The results from both the electron and muon plus jets channels were combined and compared with
different predictions of Monte Carlo generators, as well as with a set of variations in theoretical predictions due to
modelling uncertainties. In general, the results were found to be consistent with the Standard Model predictions.

The cross section measurement can be improved in a few different ways. Recent studies have shown that the binning choice
for all primary variables can be enhanced by exploiting a more sophisticated fit requiring sufficient statistics and
acceptable migration between bins. Also, addition of other variables in the template fit -- for example, the M3
variable, i.e.\ the invariant mass of the three jets yielding the highest vectorial sum of their transverse momenta --
can improve the differentiating power between the QCD and \W/\ZpJets backgrounds.

The normalised differential cross section measurement based on \SI{7}{\TeV} data \autocite{xsection_PAS_7TeV} (not
covered in this thesis) was only performed with respect to the missing transverse energy. Currently, work is under way
to improve the \SI{7}{\TeV} measurement by including other event-level variables and using the latest definitions of
reconstructed objects. This would allow the measurement of the ratio of \SI{8}{\TeV} and \SI{7}{\TeV} cross section
results, potentially cancelling more systematic uncertainties and providing more sensitivity to deviations from the
Standard Model, as well as increasing the discriminating power between MC generators. The results are due to be
published later this year, upon the completion of the current analysis work. All these differential cross section
measurements will be enhanced using 2015 LHC data at a centre of mass energy of \SI{13}{\TeV}.

The measurements shown in this thesis represent the author's contribution to the high-statistics precision measurements
of top quark physics currently being made at the LHC. Such measurements provide detailed checks of the Standard Model as
well as improved background estimates in searches for new physics signatures.


%%% ----------------------------------------------------------------------

% ------------------------------------------------------------------------

%%% Local Variables: 
%%% mode: latex
%%% TeX-master: "../thesis"
%%% End: 
