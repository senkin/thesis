%!TEX root = ../thesis.tex

\chapter[Differential cross section measurement]{Differential cross section measurement}
\label{c:xsection_analysis}
\ifpdf
    \graphicspath{{06_Cross_section_analysis/plots/}}
\else
    \graphicspath{{06_Cross_section_analysis/plots/EPS/}{06_Cross_section_analysis/plots/}}
\fi

% ------------------------------------------------------------------------
Measurement of the differential cross section of the top quark pairs with respect to different variables is an important
precision measurement, but it can also give hints of new physics. For example, measurement of the missing energy in
\ttbar events can represent a search for invisible particles produced in association with top quarks.

\section{Data and Simulation}
\label{s_xsection:data_and_simulation}

\subsection{Data}
\label{ss_xsection:data}
This analysis uses the full 2012 dataset collected by the CMS detector at a centre of mass energy of \SI{8}{\TeV}, with
a total integrated luminosity of \SI{19.7}{\fbinv}. Only certified events were used in the data, i.e.\ from such
preriods of data-taking when all of the detector subsystems were functioning with no errors. Depending on the channel,
the data were preselected with the single electron or single muon trigger. The preselection procedure as well as full
event selection will be described in Section~\ref{s_xsection:event_selection}.

\subsection{Monte Carlo samples}
\label{ss_xsection:MC_samples}
The Monte Carlo generators used in this work were presented in Section~\ref{sss_top_mass:MC_generators}. The list of
signal and background MC samples is shown in Table~\ref{tab:xsection_mc_samples}, and is largely similar to that from
the top quark mass analysis. In addition, the signal \ttjets sample is also available with \POWHEG and \MCATNLO
generators in order to be able to differentiate between them. Moreover, to extend the statistics of the \W/\ZpJets
samples, they were generated in four exclusive jet multiplicity bins: \W/\Z boson plus one/two/three and at least four
jets.

Table~\ref{tab:xsection_electron_qcd_samples} presents the list of QCD multi-jet background and $\gamma$ + jets samples
used in the estimation of the QCD background in the electron plus jets channel. The analogous set of muon-enriched QCD
samples used in the muon plus jets channel is shown in Table~\ref{tab:xsection_muon_qcd_samples}. Although the QCD
background is estimated using data-driven techniques in both channels (see Section~\ref{s_xsection:data_driven_QCD}),
the MC simulation is still used for normalisation purposes.
% Mention the systematic samples (Table~\ref{tab:xsection_systematic_samples})

\begin{table}[!htbp]
\centering
\begin{tabular}{|l|l|r|r|r|}
\toprule
Process & Generator & $\sigma$ (\pb) & \# events & $\int\lumi dt$ (\fbinv)\\
\midrule
\ttjets, \mtop = \SI{172.5}{\GeV} & \MADGRAPH & 245.8 & 6854416 & 27.9\\
\ttjets, \mtop = \SI{172.5}{\GeV} & \POWHEG  & 245.8 & 21675970 & 88.2\\
\ttjets, \mtop = \SI{172.5}{\GeV} & \MCATNLO & 245.8 & 32706581 & 133.1\\
\midrule
\WpJets ($\W \rightarrow l\nu$) & \MADGRAPH & & & \\
\hspace{5 mm}\W + 1 jet & & 5400.0 & 23141598 & 4.3 \\
\hspace{5 mm}\W + 2 jet & & 1750.0 & 34044921 & 19.5 \\
\hspace{5 mm}\W + 3 jet & & 519.0 & 15539503 & 29.9 \\
\hspace{5 mm}\W + 4 jet & & 214.0 & 13349346 & 62.4 \\
\midrule
$\Z/\gamma^* \rightarrow l^+l^- $ + jets, $m(ll) > \SI{50}{\GeV}$ & \MADGRAPH & & & \\
\hspace{5 mm}\Z + 1 jet & & 561.0 & 24045248 & 42.9 \\
\hspace{5 mm}\Z + 2 jet & & 181.0 & 21852156 & 120.7 \\
\hspace{5 mm}\Z + 3 jet & & 51.1 & 11015445 & 215.6 \\
\hspace{5 mm}\Z + 4 jet & & 23.04 & 6402827 & 277.9 \\
\midrule
Single top & \POWHEG & & & \\
\hspace{5 mm} top t-channel & & 55.5 & 3758227 & 67.7 \\
\hspace{5 mm} anti-top t-channel & & 30.0 & 1935072 & 64.5 \\
\hspace{5 mm} top s-channel & & 3.89 & 259961 & 66.8 \\
\hspace{5 mm} anti-top s-channel & & 1.76 & 139974 & 79.5 \\
\hspace{5 mm} top tW-channel & & 11.18 & 497658 & 44.5 \\
\hspace{5 mm} anti-top tW-channel & & 11.18 & 493460 & 44.1 \\
\bottomrule
\end{tabular}
\caption{Signal and background Monte Carlo samples with cross sections at $\sqrt s =
\SI{8}{\TeV}$, numbers of generated events and corresponding integrated
luminosities.}
\label{tab:xsection_mc_samples}
\end{table}

\begin{table}[!htbp] 
\centering
\resizebox{\textwidth}{!}{
\begin{tabular}{|l|l|r|r|r|r|}
\toprule
Process & Generator & $\sigma$ (\pb) & filter efficiency & \# events & $\int\lumi dt$ (\fbinv)\\
\midrule
QCD ($e/\gamma$ enriched)  & \PYTHIA & & & & \\
\hspace{5 mm}\SIrange[range-phrase = $~<\pthat<~$]{20}{30}{\GeV} 	& & \num{2.886d8} 	& \num{1.01d-2} & 34339883 & \num{1.2d-2} \\
\hspace{5 mm}\SIrange[range-phrase = $~<\pthat<~$]{30}{80}{\GeV} 	& & \num{7.433d7} 	& \num{6.21d-2} & 32537408 & \num{7.0d-3} \\
\hspace{5 mm}\SIrange[range-phrase = $~<\pthat<~$]{80}{170}{\GeV} 	& & \num{1.191d6}  	& \num{0.154} &  34542763 & \num{0.19} \\
\hspace{5 mm}\SIrange[range-phrase = $~<\pthat<~$]{170}{250}{\GeV} 	& & \num{30990.0} 	& \num{0.148} & 22862259 & \num{5.0} \\
\hspace{5 mm}\SIrange[range-phrase = $~<\pthat<~$]{250}{350}{\GeV} 	& & \num{4250.0} 	& \num{0.131} &  32505856 & \num{58.4} \\
\hspace{5 mm}$\pthat >$ \SI{350}{\GeV}  							& &	\num{810.0}  	& \num{0.11} &  33981105 & \num{381.4} \\
\midrule
QCD ($\cPqb/\cPqc \rightarrow e\nu$) & \PYTHIA & & & & \\
\hspace{5 mm}\SIrange[range-phrase = $~<\pthat<~$]{20}{30}{\GeV} 	& & \num{2.886d8} 	& \num{5.8d-4} & 1740229 & \num{1.0d-2} \\
\hspace{5 mm}\SIrange[range-phrase = $~<\pthat<~$]{30}{80}{\GeV} 	& & \num{7.433d7} 	& \num{2.25d-3} & 2048152 & \num{1.2d-2} \\
\hspace{5 mm}\SIrange[range-phrase = $~<\pthat<~$]{80}{170}{\GeV} 	& & \num{1.191d6}  	& \num{1.09d-2} & 1945525 & \num{0.15} \\
\hspace{5 mm}\SIrange[range-phrase = $~<\pthat<~$]{170}{250}{\GeV} 	& & \num{30990.0} 	& \num{2.04d-2} & 1948112 & \num{3.1} \\
\hspace{5 mm}\SIrange[range-phrase = $~<\pthat<~$]{250}{350}{\GeV} 	& & \num{4250.0}  	& \num{2.43d-2} & 2026521 & \num{19.6} \\
\hspace{5 mm}$\pthat >$ \SI{350}{\GeV} 								& &	\num{810.0}  	& \num{2.95d-2} & 1948532 & \num{81.5} \\
\midrule
$\gamma$ + jets & \MADGRAPH & & & & \\
\hspace{5 mm}\SIrange[range-phrase = $~<\HT<~$]{200}{400}{\GeV} & & \num{960.5} & 1 & 10479625 & \num{10.9} \\
\hspace{5 mm}$\HT >$ \SI{400}{\GeV} & & \num{107.5} & 1 & 1611963 & \num{15.0} \\
\bottomrule
\end{tabular}
}
\caption{QCD multi-jet background and $\gamma$ + jets MC samples used in the electron plus jets channel
with cross sections at $\sqrt s = \SI{8}{\TeV}$, numbers of generated events and corresponding integrated luminosities.
% EM-enriched samples are preselected to include jets with higher electromagnetic content;
% $\cPqb/\cPqc \rightarrow e\nu$ samples are preselected to include leptonic ($e\nu$) in-flight decays of b- and c-quarks.
}
\label{tab:xsection_electron_qcd_samples}
\end{table}

\begin{table}[!htbp] 
\centering
\resizebox{\textwidth}{!}{
\begin{tabular}{|l|l|r|r|r|r|}
\toprule
Process & Generator & $\sigma$ (\pb) & filter efficiency & \# events & $\int\lumi dt$ (\fbinv)\\
\midrule
QCD ($\mu$ enriched)  & \PYTHIA & & & & \\
\hspace{5 mm}\SIrange[range-phrase = $~<\pthat<~$]{15}{20}{\GeV} 	& & \num{7.022d8} 	& \num{3.9d-3}	& 1722681	& \num{6.3d-4} \\
\hspace{5 mm}\SIrange[range-phrase = $~<\pthat<~$]{20}{30}{\GeV} 	& & \num{2.87d8} 	& \num{6.5d-3}	& 8486904	& \num{4.5d-3} \\
\hspace{5 mm}\SIrange[range-phrase = $~<\pthat<~$]{30}{50}{\GeV} 	& & \num{6.609d7} 	& \num{1.22d-2}	& 9560265	& \num{1.2d-2} \\
\hspace{5 mm}\SIrange[range-phrase = $~<\pthat<~$]{50}{80}{\GeV} 	& & \num{8.082d6} 	& \num{2.18d-2}	& 10365230	& \num{5.9d-2} \\
\hspace{5 mm}\SIrange[range-phrase = $~<\pthat<~$]{80}{120}{\GeV} 	& & \num{1.024d6}	& \num{3.95d-2}	& 9238642	& \num{0.23} \\
\hspace{5 mm}\SIrange[range-phrase = $~<\pthat<~$]{120}{170}{\GeV} 	& & \num{1.578d5}	& \num{4.73d-2}	& 8501935	& \num{1.1} \\
\hspace{5 mm}\SIrange[range-phrase = $~<\pthat<~$]{170}{300}{\GeV} 	& & \num{34020.0}	& \num{6.76d-2}	& 7669947	& \num{3.3} \\
\hspace{5 mm}\SIrange[range-phrase = $~<\pthat<~$]{300}{470}{\GeV} 	& & \num{1757.0}	& \num{8.64d-2}	& 7832261	& \num{51.6} \\
\hspace{5 mm}\SIrange[range-phrase = $~<\pthat<~$]{470}{600}{\GeV} 	& & \num{115.2}		& \num{0.102}	& 3783069	& \num{322.0} \\
\hspace{5 mm}\SIrange[range-phrase = $~<\pthat<~$]{600}{800}{\GeV} 	& & \num{27.01}		& \num{0.0996}	& 4119000	& \num{1531.1} \\
\hspace{5 mm}\SIrange[range-phrase = $~<\pthat<~$]{800}{1000}{\GeV} & & \num{3.57}		& \num{0.1033}	& 4107853	& \num{11139.0} \\
\hspace{5 mm}$\pthat >$ \SI{1000}{\GeV}  							& &	\num{0.774}		& \num{0.1097}	& 3873970	& \num{45625.6} \\
\bottomrule
\end{tabular}
}
\caption{QCD multi-jet background MC samples used in the muon plus jets channel with cross sections at $\sqrt s =
\SI{8}{\TeV}$, numbers of generated events and corresponding integrated luminosities.}
\label{tab:xsection_muon_qcd_samples}
\end{table}

\begin{table}[!htbp]
\centering
\begin{tabular}{|l|l|r|r|r|}
\toprule
Process & Generator & $\sigma$ (\pb) & \# events & $\int\lumi dt$ (\fbinv)\\
\midrule
\ttjets & \MADGRAPH & & & \\
\hspace{5 mm}$0.5~\times$ matching threshold 	& & 245.8 & 5476728	& 10.2 \\
\hspace{5 mm}$2~\times$ matching threshold  	& & 245.8 & 5306710	& 25.5 \\
\hspace{5 mm}$0.5\times Q$  					& & 245.8 & 5387181 & 25.4 \\
\hspace{5 mm}$2\times Q$ 						& & 245.8 & 5009488 & 23.4 \\
\midrule
\WpJets ($\W \rightarrow l\nu$) & \MADGRAPH & & & \\
\hspace{5 mm}$0.5~\times$ matching threshold 	& & 29690 & 21364637 & 0.7 \\
\hspace{5 mm}$2~\times$ matching threshold 		& & 30290 & 20976082 & 0.7 \\
\hspace{5 mm}$0.5 \times Q$ 					& & 33300 & 20719363 & 0.6 \\
\hspace{5 mm}$2 \times Q$ 						& & 32000 & 20784770 & 0.6 \\
\midrule
\ZpJets ($\Z \rightarrow ll$) & \MADGRAPH & & & \\
\hspace{5 mm}$0.5~\times$ matching threshold 	& & 2888 & 2112387 & 0.6 \\
\hspace{5 mm}$2~\times$ matching threshold 		& & 2915 & 1985529 & 0.7 \\
\hspace{5 mm}$0.5 \times Q$ 					& & 3312 & 1934901 & 0.6 \\
\hspace{5 mm}$2 \times Q$ 						& & 2954 & 2170270 & 0.7 \\
\bottomrule
\end{tabular}
\caption{Systematic MC samples with cross sections at $\sqrt s =
\SI{8}{\TeV}$, numbers of generated events and corresponding integrated
luminosities. Factorisation scale $Q$ and matching threshold systematic
uncertainties are estimated with variations of \ttjets, \WpJets and \ZpJets
samples.}
\label{tab:xsection_systematic_samples}
\end{table}

% WARNING: V+Jets systematic samples cross sections are 7 TeV ones (taken from PREP)
% Does this affect the measurement of the systematics?

\subsection{Pile-up reweighting}
\label{sss_xsection:pileup_reweighting}
%EDIT%
% The number of simulated pile-up interactions does not represent the distribution of the number of pile-up events in data
% and therefore a reweighting is performed using the estimated number of in-time interactions per luminosity block. A
% centrally provided tool uses the estimated instantaneous luminosity to produce a distribution of the expected number of
% pile-up interactions. This is done for the used runs of the 2012 data taking. A weight for each number of pile-up
% interactions in the simulation are produced, taking into account the true number of pile-up interactions at the nominal
% bunch crossing.

\begin{figure}[!htpb]
\begin{center}
	\subfloat[]{\includegraphics[width=0.50\textwidth]{vertices/EPlusJets_nVertex.pdf}}\hfill
	\subfloat[]{\includegraphics[width=0.50\textwidth]{vertices/EPlusJets_nVertex_reweighted.pdf}} \\
	\subfloat[]{\includegraphics[width=0.50\textwidth]{vertices/MuPlusJets_nVertex.pdf}}\hfill
	\subfloat[]{\includegraphics[width=0.50\textwidth]{vertices/MuPlusJets_nVertex_reweighted.pdf}}
	\caption{\label{fig:pileup_vertices}
    Number of reconstructed vertices per event before (left) and after pile-up reweighting (right) in the electron
    channel (top) and in the muon channel (bottom). Both data and sum of the MC samples are normalised to unit area.}
    %QCD is estimated from MC here. Possible to fix/remove?
\end{center}
\end{figure}

 \begin{figure}[hbtp]
   \centering
     \includegraphics[width=\textwidth]{vertices/PileUp_2012_truth_data.pdf}
     \caption{Number of expected vertices per event for different inelastic cross sections for the 2012 data.}
     \label{fig:pileup_truth}
 \end{figure}

\begin{figure}[!htpb]
\begin{center}
	\subfloat[]{\includegraphics[width=0.50\textwidth]{vertices/EPlusJets_nVertex_reweighted_PU_down.pdf}}\hfill
	\subfloat[]{\includegraphics[width=0.50\textwidth]{vertices/EPlusJets_nVertex_reweighted_PU_up.pdf}} \\
	\subfloat[]{\includegraphics[width=0.50\textwidth]{vertices/MuPlusJets_nVertex_reweighted_PU_down.pdf}}\hfill
	\subfloat[]{\includegraphics[width=0.50\textwidth]{vertices/MuPlusJets_nVertex_reweighted_PU_up.pdf}}
	\caption{\label{fig:pileup_vertices_variations}
    Number of reconstructed vertices per event for the $-\sigma$ (left) and the $+\sigma$ variations (right) of
    the pile-up reweighting procedure for the electron channel (top) and the muon channel (bottom).}
\end{center}
\end{figure}

%EDIT%
% The number of interactions in the data is estimated from the measured luminosity in each bunch crossing multiplied by an
% average total inelastic cross section. The two sources of error here are the luminosity uncertainty, which is currently
% 4.4\% \cite{CMS-PAS-LUM-12-001} and the uncertainty on the total inelastic cross section. A total inelastic cross
% section of $68 \pm 4.5$~mb using forward calorimetry in 2011 data \cite{CMS-FWD-11-002} has been extrapolated to the
% 2012 value of 69.3~mb which has been used.

% In order to cover all of the modelling and physics aspects of the pileup simulation that have not been properly studied
% an additional uncertainty is recommended leading to a total uncertainty of $\pm 5\%$.

% %This method is chosen to allow for different vertex reconstruction efficiencies in data compared to the simulation.
% Figure~\ref{fig:pileup_vertices} shows the number of reconstructed vertices before and after reweighting. One can see
% that after performing the reweighting the distributions agree reasonably well for the signal region of this analysis.
% The effect of the $5\%$ change of the inelastic cross section that is used to calculate the true vertex multiplicity in
% data (fig. \ref{fig:pileup_truth}) can be seen in figure~\ref{fig:pileup_vertices_variations}. The pile-up distribution
% using the $-5\%$ value of the cross section has the best agreement between data and simulation.

\subsection{b-tagging corrections}
\label{ss_xsection:btagging_corrections}
%EDIT
% In this analysis $b$ tagging is used in the event reconstruction to associate reconstructed jets to partons in the decay
% chain and to select events in the final analysis. Here, the \emph{combined secondary vertex} tagger with the medium
% operation point  (cut value~0.679, CSVM) is used which results in a $1\%$ light-jet mistag rate ~\cite{CMSbAlgo}.

% We apply \pt dependent b-tag efficiency corrections to each $b$ tag for simulated events to account for the differences
% between data and simulation seen in 2012 data~\cite{bTagResultsPaper}. Simulated events passing the $b$ tagging
% requirements are given a weight (following the prescription from the BTag POG~\cite{bTagWeight}) to ensure that the
% probability of selecting a simulated event matches the probability of a data event with the same jet flavour composition
% being selected.

% The results of the reweighting can be seen in fig. \ref{fig:bjet_weights}. The number of MC events in the 0 and 1 b-tag
% is scaled up while the number of MC events with b-tag multiplicity higher than 2 are scaled down.


\begin{figure}[!htpb]
\begin{center}
	\subfloat[]{\includegraphics[width=0.50\textwidth]{bjets_multiplicity/EPlusJets_N_BJets.pdf}}\hfill
	\subfloat[]{\includegraphics[width=0.50\textwidth]{bjets_multiplicity/EPlusJets_N_BJets_reweighted.pdf}} \\
	\subfloat[]{\includegraphics[width=0.50\textwidth]{bjets_multiplicity/MuPlusJets_N_BJets.pdf}}\hfill
	\subfloat[]{\includegraphics[width=0.50\textwidth]{bjets_multiplicity/MuPlusJets_N_BJets_reweighted.pdf}}
	\caption{\label{fig:bjet_weights}
    b-tag multiplicity before (left) and after applying b-tag scale factors (right) in the electron channel (top) and
    the muon channel (bottom).}
\end{center}
\end{figure}

% The systematic variations of the uncertainties due to the $b$ tag efficiency are applied by varying the efficiency scale
% factor $SF$ by the according $\pm 1 \sigma$ variations in the event reweighting procedure.

\section{Event Selection}
\label{s_xsection:event_selection}
As in the top quark mass analysis, the event selection in this work is based on the standard CMS selection optimised for
the Standard Model \ttbar production. It also exploits the semi-leptonic signature of a \ttbar decay with exactly one
isolated lepton and at least four jets, two of which are b-tagged. One of the main goals of the analysis is the
measurement of the missing transverse energy distribution represented by a neutrino from the leptonic decay of the \W
boson, however, there is no specific requirement on the \MET in the selection.

Comparing to the 2011 event selection described in Section~\ref{s_top_mass:event_selection}, enhanced pile-up
subtraction and lepton identification techniques are used. Improved understanding of the detector in form of updated
alignment and calibration constants resulted in better resolution of jets and \MET. The preselection (or skimming) step
used in order to reduce the number of events for local analysis processing is identical to that of 2011 analysis.
However, additional filters are used to reject the events with artificial \MET caused by noise or known problems with
one of the detector subsystems. These filters are discussed in Section~\ref{ss_xsection:met_filters}.

Two semi-leptonic \ttbar decay modes are explored in this analysis: the electron plus jets and the muon plus jets
channels. The event selection for each of these decay modes is described below.

\subsection{Electron plus jets channel}
\label{ss_xsection:ejets}

Similarly to the electron plus jets channel in the top quark mass analysis, the selection consists of the following
steps:

\begin{enumerate}[topsep=\parskip, parsep=\parskip, itemsep=\parskip, leftmargin=\leftmargin]
	\item preselection;
	\item trigger;
	\item electron candidate selection;
	\item dilepton veto;
	\item conversion veto;
	\item muon veto;
	\item jet selection;
	\item b-tagging.
\end{enumerate}

%https://twiki.cern.ch/twiki/bin/view/CMS/TWikiTop2011DataMCTrig

\subsubsection*{Trigger}
As opposed to 2011 analysis using cross triggers, here a single electron trigger is used. Referred to as
HLT\_Ele27\_WP80, this trigger has a lepton \pt threshold of \SI{27}{\GeV} and ``WP80'' lepton isolation requirement
(see Table~\ref{tab:trigger_naming}), but no specific jet requirements. In contrast to 2011 data-taking, the single
electron trigger was unprescaled for the whole period of 2012 running, despite a comparatively large rate. This decision
was made due to simplicity of the trigger selection, requiring flat efficiency corrections.

% \subsubsection*{Electron candidate selection}
% Exactly one electron candidate satisfying all the following criteria is required to be present in the event. As the
% electron-plus-jets trigger has a lepton \pt threshold of \SI{25}{\GeV}, the electron candidate transverse momentum is
% required to be above \SI{30}{\GeV}, in order to be in the trigger efficiency plateau region. Furthermore, the electron
% has to be within the tracker region of $|\eta| < 2.5$ excluding the ECAL barrel-endcap transition regions of $1.4442 <
% |\eta| < 1.566$. The CiC electron ID (see Section~\ref{ss:electron_reconstruction}) with the tightest working point
% (``HyperTight'') is used for electron identification purposes. This working point provides a misidentification rate of
% less than \SI{1}{\pc}, and overall identification efficiency of \SI{75}{\pc} \autocite{CiC_ID}.

% In order to only select electrons originating at the interaction vertex, the $x$-$y$ distance ($d_{xy}$) between the
% electron track and the interaction point (2D impact parameter) is required to be less than \SI{0.02}{\cm}. The
% contribution of electrons inside jets from QCD background events, as well as fake electrons, is reduced by imposing the
% PF-based relative isolation cut on the variable \reliso defined in Section~\ref{sss:electron_isolation}. A $\Delta
% R$-cone with size of \num{0.3} is used in calculation of the \reliso variable, and events with \reliso $< 0.1$ are
% accepted.

% \subsubsection*{Dilepton veto}
% A veto requirement on a second electron candidate (or dilepton veto) is used to reject events with any additional
% electrons. Looser criteria are used to identify the second electron in an event. Namely, the event is rejected if there
% is a second electron satisfying a lower \ET threshold of \SI{15}{\GeV} and a looser cut on \reliso ($<0.2$).

% Additionally, in order to reduce \ZpJets background, the following veto is applied. If the event contains a second
% electron with $\ET > \SI{30}{\GeV}$ and $\reliso < 1.0$ that forms such an invariant mass with the signal electron
% candidate that it is close to the \Z mass peak, it is also rejected. The invariant mass window around the \Z peak is
% chosen to be $\SI{76}{\GeV} < m_{ee} < \SI{106}{\GeV}$.


% \subsubsection*{Conversion veto}
% As mentioned in Section~\ref{sss:photon_conversions}, two methods are used in this analysis to remove electrons coming
% from photon conversions: missing pixel layers method and partner track matching method. If missing pixel layer hits or a
% matching partner track are present in the event, it is rejected.

% \subsubsection*{Muon veto}
% Other \ttbar channels, including muon and dilepton decay modes, can contaminate events passing the electron plus jets
% selection. To reduce this contamination, events containing an isolated global muon (see Section
% \ref{ss:muon_reconstruction}) are rejected. The muon is required to have a \pt above \SI{10}{\GeV}, $|\eta| < 2.5$ and
% $\reliso < 0.2$ with a $\Delta R$ cone of \num{0.4}.

% \subsubsection*{Jet selection and b-tagging}
% Final steps in the selection help to further reduce the background by imposing constraints on the number of jets. This
% is particularly effective to mitigate the \WpJets contamination, as the number of \WpJets events decreases exponentially
% with increasing number of jets. In this analysis, at least four jets with $\pt > \SI{30}{\GeV}$ and $|\eta| < 2.4$ are
% required to be present in the event. These jets have to pass the loose PF jet ID (see Section
% \ref{ss:jet_reconstruction}). Finally, the CSV b-tagging algorithm with medium working point (Section
% \ref{sss:b-tagging}) is used to identify the two \cPqb-quarks from the \ttbar decay.

\subsection{Muon plus jets channel}
\label{ss_xsection:mujets}

% \begin{figure}[!htp]
%    \centering
%    \subfloat[]{\includegraphics[width=0.5\textwidth]{placeholder}}
%    \subfloat[]{\includegraphics[width=0.5\textwidth]{placeholder}}
%    \hfill
%    \subfloat[]{\includegraphics[width=0.5\textwidth]{placeholder}}
%    \subfloat[]{\includegraphics[width=0.5\textwidth]{placeholder}}
%    \caption{\label{fig:controlplots}Kinematic variable distributions after all selection cuts. (a) electron \pt, (b)
%    electron $\eta$, (c) jet \pt for all jets passing the selection, (d) \MET.}
% \end{figure}

\subsection{Additional filters}
\label{ss_xsection:met_filters}
%https://twiki.cern.ch/twiki/bin/viewauth/CMS/MissingETOptionalFilters

A set of optional filters, prescribed for analyses with high sensitivity to missing transverse energy, was applied in
the selection for both electron and muon channels. Most of these filters were put in place after discovering events with
anomalously high \MET. A short description of each filter is given below.

\begin{description}[wide=\parindent]
	\item[CSC beam halo filter.]
	Protons in the LHC beams can interact with residual particles of beam collimators, which produces secondary particle
	showers refered to as beam halo \autocite{beam_halo_CMS}. This machine-induced background can significantly
	contaminate collision events, particularly affecting the \MET observable due to specific trajectories of halo muons.
	A dedicated algorithm based on exploiting the halo signatures in the Cathode Strip Chambers (CSCs) helps to mitigate
	this background.

	\item[HCAL laser filter.] The HCAL laser system is used for calibration and monitoring the detector response
	\autocite{CMS_TDR1}. During the 2012 data-taking, laser pulses were accidentally fired and consequently polluted a
	small fraction of the recorded physics dataset. Using the fact that unlike physics events, laser pulses produce hits
	in calibration channels of the HCAL, the dataset is filtered from this contamination.
	%https://indico.cern.ch/event/169318/contribution/1/material/slides/0.pdf

	\item[ECAL dead cell filter.] Both ECAL endcap and barrel regions are known to have faulty crystals producing too
	much noise in detector readouts. There are also crystals corresponding to front-end cards with defective data link.
	All these channels are masked in reconstruction and constitute to about \SI{1}{\pc} of the total amount. However,
	they can still cause a significant amount of energy deposits to be lost, therefore contributing to fake \MET. One of
	the approaches to tackle this issue uses the so-called trigger primitive information, i.e.\ digital quantities
	produced by the Level 1 trigger electronics \autocite{CMS_L1_Trigger_TDR}. Another approach uses the energy deposits
	surrounding the masked channels (boundary energy). These methods allow to filter events with the estimated energy
	leakage above certain threshold (\SI{\sim10}{\GeV}).
	%https://twiki.cern.ch/twiki/bin/viewauth/CMS/SusyEcalMaskedCellSummary

	\item[Tracking failure filter.] Some events have been observed to have a lack of tracks corresponding to standard or
	large calorimeter deposits. There are two understood sources of this misreconstruction: in a first type of such
	events, the tracking algorithm (explained in Section~\ref{ss:electron_reconstruction}) halts for some of its
	iterations due to an exceeding number of calorimeter clusters. In a second type the hard collision happens at a
	distance of approximately \SI{75}{\cm} from the nominal interaction point. A cut on the summed \pt of the tracks
	originating from the primary vertices passing the preselection criteria, divided by the summed \pt of all jets in
	the event, was found to clearly distinguish between pathological and unaffected events. A cut value of \SI{10}{\pc}
	is imposed.
	%https://twiki.cern.ch/twiki/bin/viewauth/CMS/SusyRA2NJetsInData2010#Tracking_failure

	\item[Bad ECAL endcap super-cluster filter.] In 2012, two $5\times5$ super-clusters in the ECAL endcap regions were
	observed to occasionally produce anomalous pulses, leading to artificially high \MET. The source of this issue can
	be related to the High-Voltage system, but is subject to further investigation. A filter removing the problematic
	events is used, configured to cut on the total energy of reconstructed hits not passing the nominal quality criteria
	in the two super-clusters, with the cut value of \SI{1}{\TeV}.
	%https://twiki.cern.ch/twiki/bin/viewauth/CMS/MissingETOptionalFilters#Bad_EE_Supercrystal_filter_added

	\item[ECAL laser correction filter.] Lead tungstate crystals of the ECAL naturally lose their transparency with
	radiation exposure. To maintain high precision, this effect is corrected for by frequently injecting laser pulses
	and reading out the response in order to calibrate each crystal \autocite{CMS_TDR1}. During 2012 data-taking, a
	handful of crystals unphysically large values of laser corrections, resulting in anomalously high \MET in recorded
	events. The filter removes such events from the dataset.
	%https://twiki.cern.ch/twiki/bin/viewauth/CMS/PdmVKnowFeatures#Filter_to_reject_events_with_ano

	\item[Strip tracker noise filter.] A few events were affected by a large coherent noise in the silicon strip
	tracker, which resulted in a much larger number of the strip clusters compared to according pixel cluster
	multiplicity in the tracking algorithm. A dedicated filter rejects events with this anomaly.
	%https://twiki.cern.ch/twiki/bin/viewauth/CMS/TrackingPOGFilters#Filters

\end{description}

\section{Data-driven QCD estimation}
\label{s_xsection:data_driven_QCD}

\section{Choice of binning}
\label{s_xsection:binning}

\section{Differential cross section measurement}
\label{s_xsection:measurement}

\section{Unfolding}
\label{s_xsection:unfolding}

\section{Systematic Uncertainties}
\label{s_xsection:systematics}

\section{Results}
\label{s_xsection:results}

%%% Local Variables: 
%%% mode: latex
%%% TeX-master: "../thesis"
%%% End: 
