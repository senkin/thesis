%!TEX root = ../thesis.tex

\chapter[Differential cross section measurement]{Differential cross section measurement}
\label{c:xsection_analysis}
\ifpdf
    \graphicspath{{06_Cross_section_analysis/plots/}}
\else
    \graphicspath{{06_Cross_section_analysis/plots/EPS/}{06_Cross_section_analysis/plots/}}
\fi

% ------------------------------------------------------------------------
Measurement of the differential cross section of the top quark pairs with respect to different variables is an important
precision measurement, but it can also give hints of new physics. For example, measurement of the missing energy in
\ttbar events can represent a search for invisible particles produced in association with top quarks.

\section{Data and Simulation}
\label{s_xsection:data_and_simulation}

\subsection{Data}
\label{ss_xsection:data}
This analysis uses the full 2012 dataset collected by the CMS detector at a centre of mass energy of \SI{8}{\TeV}, with
a total integrated luminosity of \SI{19.7}{\fbinv}. Only certified events were used in the data, i.e.\ from such
preriods of data-taking when all of the detector subsystems were functioning with no errors.

\subsection{Simulation of signal and background processes}
\label{ss_xsection:signal_and_background}

\subsubsection{Monte Carlo samples}
\label{sss_xsection:MC_samples}

\begin{table}[!htbp]
\centering
\caption{Signal and background Monte Carlo samples with cross sections at $\sqrt s =
\SI{8}{\TeV}$, numbers of generated events and corresponding integrated
luminosities.}
\label{tab:xsection_mc_samples}
\begin{tabular}{|l|l|r|r|r|}
\toprule
Process & Generator & $\sigma$ (\pb) & \# events & $\int\lumi dt$ (\fbinv)\\
\midrule
\ttjets, \mtop = \SI{172.5}{\GeV} & \MADGRAPH & 245.8 & 6854416 & 27.9\\
\ttjets, \mtop = \SI{172.5}{\GeV} & \POWHEG  & 245.8 & 21675970 & 88.2\\
\ttjets, \mtop = \SI{172.5}{\GeV} & \MCATNLO & 245.8 & 32706581 & 133.1\\
\midrule
\WpJets ($\W \rightarrow l\nu$) & \MADGRAPH & & & \\
\hspace{5 mm}\W + 1 jet & & 5400.0 & 23141598 & 4.3 \\
\hspace{5 mm}\W + 2 jet & & 1750.0 & 34044921 & 19.5 \\
\hspace{5 mm}\W + 3 jet & & 519.0 & 15539503 & 29.9 \\
\hspace{5 mm}\W + 4 jet & & 214.0 & 13349346 & 62.4 \\
\midrule
$\Z/\gamma^* \rightarrow l^+l^- $ + jets, $m(ll) > \SI{50}{\GeV}$ & \MADGRAPH & & & \\
\hspace{5 mm}\Z + 1 jet & & 561.0 & 24045248 & 42.9 \\
\hspace{5 mm}\Z + 2 jet & & 181.0 & 21852156 & 120.7 \\
\hspace{5 mm}\Z + 3 jet & & 51.1 & 11015445 & 215.6 \\
\hspace{5 mm}\Z + 4 jet & & 23.04 & 6402827 & 277.9 \\
\midrule
Single top & \POWHEG & & & \\
\hspace{5 mm} top t-channel & & 55.5 & 3758227 & 67.7 \\
\hspace{5 mm} anti-top t-channel & & 30.0 & 1935072 & 64.5 \\
\hspace{5 mm} top s-channel & & 3.89 & 259961 & 66.8 \\
\hspace{5 mm} anti-top s-channel & & 1.76 & 139974 & 79.5 \\
\hspace{5 mm} top tW-channel & & 11.18 & 497658 & 44.5 \\
\hspace{5 mm} anti-top tW-channel & & 11.18 & 493460 & 44.1 \\
\bottomrule
\end{tabular}
\end{table}

\begin{table}[!htbp] 
\centering
\caption{QCD multi-jet background and $\gamma$ + jets MC samples used in the electron plus jets channel
with cross sections at $\sqrt s = \SI{8}{\TeV}$, numbers of generated events and corresponding integrated luminosities.
% EM-enriched samples are preselected to include jets with higher electromagnetic content;
% $\cPqb/\cPqc \rightarrow e\nu$ samples are preselected to include leptonic ($e\nu$) in-flight decays of b- and c-quarks.
}
\label{tab:xsection_electron_qcd_samples}
\resizebox{\textwidth}{!}{
\begin{tabular}{|l|l|r|r|r|r|}
\toprule
Process & Generator & $\sigma$ (\pb) & filter efficiency & \# events & $\int\lumi dt$ (\fbinv)\\
\midrule
QCD ($e/\gamma$ enriched)  & \PYTHIA & & & & \\
\hspace{5 mm}\SIrange[range-phrase = $~<\pthat<~$]{20}{30}{\GeV} 	& & \num{2.886d8} 	& \num{1.01d-2} & 34339883 & \num{1.2d-2} \\
\hspace{5 mm}\SIrange[range-phrase = $~<\pthat<~$]{30}{80}{\GeV} 	& & \num{7.433d7} 	& \num{6.21d-2} & 32537408 & \num{7.0d-3} \\
\hspace{5 mm}\SIrange[range-phrase = $~<\pthat<~$]{80}{170}{\GeV} 	& & \num{1.191d6}  	& \num{0.154} &  34542763 & \num{0.19} \\
\hspace{5 mm}\SIrange[range-phrase = $~<\pthat<~$]{170}{250}{\GeV} 	& & \num{30990.0} 	& \num{0.148} & 22862259 & \num{5.0} \\
\hspace{5 mm}\SIrange[range-phrase = $~<\pthat<~$]{250}{350}{\GeV} 	& & \num{4250.0} 	& \num{0.131} &  32505856 & \num{58.4} \\
\hspace{5 mm}$\pthat >$ \SI{350}{\GeV}  							& &	\num{810.0}  	& \num{0.11} &  33981105 & \num{381.4} \\
\midrule
QCD ($\cPqb/\cPqc \rightarrow e\nu$) & \PYTHIA & & & & \\
\hspace{5 mm}\SIrange[range-phrase = $~<\pthat<~$]{20}{30}{\GeV} 	& & \num{2.886d8} 	& \num{5.8d-4} & 1740229 & \num{1.0d-2} \\
\hspace{5 mm}\SIrange[range-phrase = $~<\pthat<~$]{30}{80}{\GeV} 	& & \num{7.433d7} 	& \num{2.25d-3} & 2048152 & \num{1.2d-2} \\
\hspace{5 mm}\SIrange[range-phrase = $~<\pthat<~$]{80}{170}{\GeV} 	& & \num{1.191d6}  	& \num{1.09d-2} & 1945525 & \num{0.15} \\
\hspace{5 mm}\SIrange[range-phrase = $~<\pthat<~$]{170}{250}{\GeV} 	& & \num{30990.0} 	& \num{2.04d-2} & 1948112 & \num{3.1} \\
\hspace{5 mm}\SIrange[range-phrase = $~<\pthat<~$]{250}{350}{\GeV} 	& & \num{4250.0}  	& \num{2.43d-2} & 2026521 & \num{19.6} \\
\hspace{5 mm}$\pthat >$ \SI{350}{\GeV} 								& &	\num{810.0}  	& \num{2.95d-2} & 1948532 & \num{81.5} \\
\midrule
$\gamma$ + jets & \MADGRAPH & & & & \\
\hspace{5 mm}\SIrange[range-phrase = $~<\HT<~$]{200}{400}{\GeV} & & \num{960.5} & 1 & 10479625 & \num{10.9} \\
\hspace{5 mm}$\HT >$ \SI{400}{\GeV} & & \num{107.5} & 1 & 1611963 & \num{15.0} \\
\bottomrule
\end{tabular}
}
\end{table}

\begin{table}[!htbp] 
\centering
\caption{QCD multi-jet background MC samples used in the muon plus jets channel with cross sections at $\sqrt s =
\SI{8}{\TeV}$, numbers of generated events and corresponding integrated luminosities.}
\label{tab:xsection_muon_qcd_samples}
\resizebox{\textwidth}{!}{
\begin{tabular}{|l|l|r|r|r|r|}
\toprule
Process & Generator & $\sigma$ (\pb) & filter efficiency & \# events & $\int\lumi dt$ (\fbinv)\\
\midrule
QCD ($\mu$ enriched)  & \PYTHIA & & & & \\
\hspace{5 mm}\SIrange[range-phrase = $~<\pthat<~$]{15}{20}{\GeV} 	& & \num{7.022d8} 	& \num{3.9d-3}	& 1722681	& \num{6.3d-4} \\
\hspace{5 mm}\SIrange[range-phrase = $~<\pthat<~$]{20}{30}{\GeV} 	& & \num{2.87d8} 	& \num{6.5d-3}	& 8486904	& \num{4.5d-3} \\
\hspace{5 mm}\SIrange[range-phrase = $~<\pthat<~$]{30}{50}{\GeV} 	& & \num{6.609d7} 	& \num{1.22d-2}	& 9560265	& \num{1.2d-2} \\
\hspace{5 mm}\SIrange[range-phrase = $~<\pthat<~$]{50}{80}{\GeV} 	& & \num{8.082d6} 	& \num{2.18d-2}	& 10365230	& \num{5.9d-2} \\
\hspace{5 mm}\SIrange[range-phrase = $~<\pthat<~$]{80}{120}{\GeV} 	& & \num{1.024d6}	& \num{3.95d-2}	& 9238642	& \num{0.23} \\
\hspace{5 mm}\SIrange[range-phrase = $~<\pthat<~$]{120}{170}{\GeV} 	& & \num{1.578d5}	& \num{4.73d-2}	& 8501935	& \num{1.1} \\
\hspace{5 mm}\SIrange[range-phrase = $~<\pthat<~$]{170}{300}{\GeV} 	& & \num{34020.0}	& \num{6.76d-2}	& 7669947	& \num{3.3} \\
\hspace{5 mm}\SIrange[range-phrase = $~<\pthat<~$]{300}{470}{\GeV} 	& & \num{1757.0}	& \num{8.64d-2}	& 7832261	& \num{51.6} \\
\hspace{5 mm}\SIrange[range-phrase = $~<\pthat<~$]{470}{600}{\GeV} 	& & \num{115.2}		& \num{0.102}	& 3783069	& \num{322.0} \\
\hspace{5 mm}\SIrange[range-phrase = $~<\pthat<~$]{600}{800}{\GeV} 	& & \num{27.01}		& \num{0.0996}	& 4119000	& \num{1531.1} \\
\hspace{5 mm}\SIrange[range-phrase = $~<\pthat<~$]{800}{1000}{\GeV} & & \num{3.57}		& \num{0.1033}	& 4107853	& \num{11139.0} \\
\hspace{5 mm}$\pthat >$ \SI{1000}{\GeV}  							& &	\num{0.774}		& \num{0.1097}	& 3873970	& \num{45625.6} \\
\bottomrule
\end{tabular}
}
\end{table}

\begin{table}[!htbp]
\centering
\caption{Systematic MC samples with cross sections at $\sqrt s =
\SI{8}{\TeV}$, numbers of generated events and corresponding integrated
luminosities. Factorisation scale $Q$ and matching threshold systematic
uncertainties are estimated with variations of \ttjets, \WpJets and \ZpJets
samples.}
\label{tab:xsection_systematic_samples}
\begin{tabular}{|l|l|r|r|r|}
\toprule
Process & Generator & $\sigma$ (\pb) & \# events & $\int\lumi dt$ (\fbinv)\\
\midrule
\ttjets & \MADGRAPH & & & \\
\hspace{5 mm}$0.5~\times$ matching threshold 	& & 245.8 & 5476728	& 10.2 \\
\hspace{5 mm}$2~\times$ matching threshold  	& & 245.8 & 5306710	& 25.5 \\
\hspace{5 mm}$0.5\times Q$  					& & 245.8 & 5387181 & 25.4 \\
\hspace{5 mm}$2\times Q$ 						& & 245.8 & 5009488 & 23.4 \\
\midrule
\WpJets ($\W \rightarrow l\nu$) & \MADGRAPH & & & \\
\hspace{5 mm}$0.5~\times$ matching threshold 	& & 29690 & 21364637 & 0.7 \\
\hspace{5 mm}$2~\times$ matching threshold 		& & 30290 & 20976082 & 0.7 \\
\hspace{5 mm}$0.5 \times Q$ 					& & 33300 & 20719363 & 0.6 \\
\hspace{5 mm}$2 \times Q$ 						& & 32000 & 20784770 & 0.6 \\
\midrule
\ZpJets ($\Z \rightarrow ll$) & \MADGRAPH & & & \\
\hspace{5 mm}$0.5~\times$ matching threshold 	& & 2888 & 2112387 & 0.6 \\
\hspace{5 mm}$2~\times$ matching threshold 		& & 2915 & 1985529 & 0.7 \\
\hspace{5 mm}$0.5 \times Q$ 					& & 3312 & 1934901 & 0.6 \\
\hspace{5 mm}$2 \times Q$ 						& & 2954 & 2170270 & 0.7 \\
\bottomrule
\end{tabular}
\end{table}

% WARNING: V+Jets systematic samples cross sections are 7 TeV ones (taken from PREP)
% Does this affect the measurement of the systematics?

\subsubsection{Pile-up reweighting}
\label{sss_xsection:pileup_reweighting}

\subsubsection{b-tagging corrections}
\label{sss_xsection:btagging_corrections}

\section{Event Selection}
\label{s_xsection:event_selection}

\section{Choice of binning}
\label{s_xsection:binning}

\section{Differential cross section measurement}
\label{s_xsection:measurement}

\section{Unfolding}
\label{s_xsection:unfolding}

\section{Systematic Uncertainties}
\label{s_xsection:systematics}

\section{Results}
\label{s_xsection:results}

%%% Local Variables: 
%%% mode: latex
%%% TeX-master: "../thesis"
%%% End: 
