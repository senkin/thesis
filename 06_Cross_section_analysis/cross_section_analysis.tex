%!TEX root = ../thesis.tex

\chapter[Differential cross section measurement]{Differential cross section measurement}
\label{c:xsection_analysis}
\ifpdf
    \graphicspath{{06_Cross_section_analysis/plots/}}
\else
    \graphicspath{{06_Cross_section_analysis/plots/EPS/}{06_Cross_section_analysis/plots/}}
\fi

% ------------------------------------------------------------------------
Measurement of the differential cross section of the top quark pairs with respect to different variables is an important
precision measurement, but it can also give hints of new physics. For example, measurement of the missing energy in
$t\bar{t}$ events can represent a search for invisible particles produced in association with top quarks.

\section{Data and Simulation}
\label{s_xsection:data_and_simulation}

\subsection{Data}
\label{ss_xsection:data}

\subsection{Simulation of signal and background processes}
\label{ss_xsection:signal_and_background}

\subsubsection{Monte Carlo samples}
\label{sss_xsection:MC_samples}

\section{Event Selection}
\label{s_xsection:event_selection}

\section{Choice of binning}
\label{s_xsection:binning}

\section{Differential cross section measurement}
\label{s_xsection:measurement}

\section{Unfolding}
\label{s_xsection:unfolding}

\section{Systematic Uncertainties}
\label{s_xsection:systematics}

\section{Results}
\label{s_xsection:results}

%%% Local Variables: 
%%% mode: latex
%%% TeX-master: "../thesis"
%%% End: 
